%latex 数学公式编辑初步:行间公式,行内公式,特殊符号的输入,从mathtype粘贴代码,公式的编号,subequations,\newcommand,\boxed,\mathop
%\numberwithin{equation}{...}
%行内公式:在内容中显示公式,且不影响行间距
%行间公式:另起一行来显示公式,且居中显示
%\pm 表示 +-
%\frac表示分数
%\sqrt{}表示开方运算
%使用数学负号必须用$...$框选起来

\documentclass{book}    %article,report,letter
\usepackage{amsmath}

\begin{document}

\chapter{intro}
\section{history}
The quick brown fox jumps over the lazy dog.The quick brown fox jumps over the lazy dog.The quick brown fox jumps over the lazy dog.The quick brown fox jumps over the lazy dog.The quick brown
$\frac{{- b \pm \sqrt {{b^2} - 4ac} }}{2a}$ fox jumps over the lazy dog.The quick brown fox jumps over the lazy dog.The quick brown fox jumps over the lazy dog.


$$\frac{{- b \pm \sqrt {{b^2} - 4ac} }}{2a}$$       %两种行间公式的表达形式     $$...$$     \[...\]
\[\frac{{- b \pm \sqrt {{b^2}  -4ac} }}{2a}\]

$$\oint{abc}$$

$$\sqrt[3]{1888}$$



\section{abcd}

\chapter{deduction}
\section{deduction 1}


\end{document} 