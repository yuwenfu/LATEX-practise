%LaTex的数学公式编辑初步2:\left & \right,矩阵的输入(bmatrix,Bmatrix,pmatrix,vmatrix,Vmatrix,array),分块矩阵
%\hdashline{2pt/2pt}
%;{2pt|2pt}

\documentclass{book}    %article,report,letter
\usepackage{amsmath}	%用于矩阵的宏包(bmatrix,Bmatrix,pmatrix,vmatrix,Vmatrix)
\usepackage{mathrsfs}
\usepackage{arydshln}	%推荐,用于矩阵分块,用虚线划分 
						%竖线命令如;{4pt|2pt},即实线4pt长,虚线2pt长
						%横线命令如\hdashline
						%pmat第三方宏包,也是用于虚线划分虚线

\newcommand{\fc}{\frac}

\begin{document}
	
	$$lim_{n\to\infty}(1+\fc{1}{n})^n=e$$ 
	 %存在问题:括号非常小
	$$lim_{n\to\infty}\left(1+\fc{1}{n}\right)^n=e$$ %解决方法
	%\left & \right设置长括号
		
	$$\frac{\partial y}{\partial x}|^{y_2=4}_{y_1=1}$$ 
	%存在问题:竖线非常短
	$$\left.\frac{\partial y}{\partial x}\right|^{y_2=4}_{y_1=1}$$ 
	%解决方法:\left和\frac中间的点是分隔符
	
	%array用于矩阵的排版
	$$
	\begin{array}{ccc}  %括号中ccc表示列对齐方式
	222 & 333 & 78 \\ 
	111 & 666 & 45 \\ 
	56 & 78 & 68
	\end{array} 
	$$
	$$
	\left(
	\begin{array}{clr}
	222 & 333 & 78 \\ 
	111 & 666 & 45 \\ 
	56 & 78 & 68
	\end{array} 
	\right)
	$$
	$$
	\left[
	\begin{array}{ccc}
	222 & 333 & 78 \\ 
	111 & 666 & 45 \\ 
	56 & 78 & 68
	\end{array} 
	\right]
	$$	
	
	%\usepackage{amsmath}  矩阵的宏包(bmatrix,Bmatrix,pmatrix,vmatrix,Vmatrix,array)	
	$$
	\begin{bmatrix}   %不用写ccc对齐方式,bmatrix的作用:为矩阵增加中括号
	222 & 333 & 78 \\ 
	111 & 666 & 45 \\ 
	56 & 78 & 68
	\end{bmatrix}
	$$
	
	$$
	\begin{Bmatrix}   %Bmatrix的作用:为矩阵增加大括号
	222 & 333 & 78 \\ 
	111 & 666 & 45 \\ 
	56 & 78 & 68
	\end{Bmatrix}
	$$
	
	$$
	\begin{vmatrix}   %vmatrix的作用:为矩阵增加竖线,即行列式
	222 & 333 & 78 \\ 
	111 & 666 & 45 \\ 
	56 & 78 & 68
	\end{vmatrix}
	$$
	
	$$
	\begin{Vmatrix}   %Vmatrix的作用:为矩阵增加双竖线
	222 & 333 & 78 \\ 
	111 & 666 & 45 \\ 
	56 & 78 & 68
	\end{Vmatrix}
	$$
	
	$$
	\begin{pmatrix}   %pmatrix的作用:为矩阵增加括号,即矩阵
	222 & 333 & 78 \\ 
	111 & 666 & 45 \\ 
	56 & 78 & 68
	\end{pmatrix}
	$$
	
	%矩阵分块
	$$
	\left[
	\begin{array}{c|cc}
	222 & 333 & 78 \\ 
	\hline
	111 & 666 & 45 \\ 
	56 & 78 & 68
	\end{array} 
	\right]
	$$	
	
	%\usepackage{arydshln}	%用于矩阵分块,用虚线划分	
	$$
	\left[
	\begin{array}{c|c;{4pt/2pt}c}
	222 & 333 & 78 \\ 
	\hline
	111 & 666 & 45 \\ 
	\hdashline{4pt/2pt}
	56 & 78 & 68
	\end{array} 
	\right]
	$$
	
		
\end{document}