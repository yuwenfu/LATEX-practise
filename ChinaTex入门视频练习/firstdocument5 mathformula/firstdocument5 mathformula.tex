%LaTex的数学公式编辑进阶,公式的对齐方式(split、align、gathered、eqnarray),\nonumber,casee环境

%数学公式排版书籍:More Math into LaTex --Springer
%ChinaTex 的新浪微博

\documentclass{book}

\usepackage{mathrsfs}
\usepackage{amsmath}
\usepackage{graphicx}
\numberwithin{equation}{section}  %使公式编号按照章节来编号

\begin{document}

\chapter{intro}
\[z=(a+b)^4=(a+b)^2(a+b)^2=(a^2+2ab+b^2)(a^2+2ab+b^2)=a^4+4a^3b+6a^2b^2+4ab^3+b^4\]
 
%公式对齐方法

\chapter{intro1}
\begin{equation}	%split环境缺点,只能所有公式共用同一标号
\begin{split}  %利用split环境对公式进行排版
z&=(a+b)^4=(a+b)^2(a+b)^2  \\
&=(a^2+2ab+b^2)(a^2+2ab+b^2) \\
&=a^4+4a^3b+6a^2b^2+4ab^3+b^4		
\end{split}
\end{equation}	


\begin{align}  %align环境每一行都有编号,且可以独立存在,不需要数学环境
z&=(a+b)^4=(a+b)^2(a+b)^2  \\ %以等号作为划分
&=(a^2+2ab+b^2)(a^2+2ab+b^2) \\
&=a^4+4a^3b+6a^2b^2+4ab^3+b^4		
\end{align}
	
\begin{align}  %\nonumber相应的行没有编号
z&=(a+b)^4=(a+b)^2(a+b)^2 \nonumber \\
&=(a^2+2ab+b^2)(a^2+2ab+b^2) \nonumber\\
&=a^4+4a^3b+6a^2b^2+4ab^3+b^4		
\end{align}

\begin{equation}
\begin{gathered}  %gathered环境合并公式,并共用编号,排列关于中心对称
z=(a+b)^4=(a+b)^2(a+b)^2 \\
z=(a^2+2ab+b^2)(a^2+2ab+b^2) \\
z=a^4+4a^3b+6a^2b^2+4ab^3+b^4		
\end{gathered}
\end{equation}

\chapter{intro2}	

\begin{eqnarray}
	%如需以等号划分,须在等号两端加上&...&
	%等号两端的距离并被拉大,起强调作用
z&=&(a+b)^4=(a+b)^2(a+b)^2 \nonumber \\
&=&(a^2+2ab+b^2)(a^2+2ab+b^2) \nonumber\\
&=&a^4+4a^3b+6a^2b^2+4ab^3+b^4
\end{eqnarray}

%分段函数的表示方法:cases环境
Indicator Function:
\[
I_A(a)=
\begin{cases}
	1&a \in A \\
	0&a \not\in A
\end{cases}
\]

\end{document}