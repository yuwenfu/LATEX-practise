%主讲人:Chinatex  www.chinatex.com
%数学定理的编号设置
%数学定理的说明文字加入
%数学定理的样式选择(plain,definition,remark)
%证明环境简介(proof)

\documentclass[openany]{book}  %[openany]的作用:去掉奇数页码生成空白页的问题

\usepackage{mathrsfs}
\usepackage{amsthm}		%星号*要调用此宏包;proof后面生成已证明的小符号
\usepackage{amsmath}	%美国数学协会定义的数学宏包	
\usepackage{graphicx} %插入图片的宏包

\numberwithin{equation}{section}  %使公式编号按照章节来编号

%\theoremstyle{plain}  %新的式样,有三种风格:plain、definition、remark,默认为plain
\theoremstyle{plain} 
\newtheorem{thm}{Theorem}[section] 	
							%定义定理相关的环境
							%thm和Theorem同义
							%[chapter]的作用:定理编号加上章节编号

\theoremstyle{definition}
\newtheorem{defn}[thm]{Definition} %定理定义
							%[thm]的作用:继承theorem的计数器
\theoremstyle{remark}
\newtheorem{rmk}{Remark} %文章需要注意的事项

\theoremstyle{plain} 
\newtheorem*{nt}{Note} %定理笔记  %星号的作用是去掉显示数字标号,要调用宏包amsthm

%renewcommand命令是对现有命令进行更新
\renewcommand{\proofname}{\sffamily\bfseries myproof}

\begin{document}
	\chapter{Introduction}
	
	
	\chapter{Advanced}
	\section{History}
	
	\section{Suggestion}
	
	\section{Conclusion}
	\begin{thm}[the first fundamental theorem] %方框是定理补充说明文字
		This is a Theorem env.
	\end{thm}
	
	\begin{thm}
		This is also a Theorem env.
	\end{thm}

	\begin{thm}
		This is also a Theorem env.
	\end{thm}
	
	\begin{nt}
	This is a note with no number.
	\end{nt}

	\begin{defn}
	This is a Definition env.
	\end{defn}

	\begin{defn}
	This is also a Definition env.
	\end{defn}

	\begin{rmk}
	This is a remark env.
	\end{rmk}

	\begin{rmk}
	This is a also remark env.
	\end{rmk}

	\chapter{Summary}
	\begin{proof}  %证明环境
	This is a proof env.	
	\end{proof}
	
\end{document}