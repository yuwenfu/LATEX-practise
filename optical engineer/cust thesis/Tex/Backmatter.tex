\chapter[致谢]{致\quad 谢}\chaptermark{致\quad 谢}% syntax: \chapter[目录]{标题}\chaptermark{页眉}
\thispagestyle{noheaderstyle}% 如果需要移除当前页的页眉
\pagestyle{noheaderstyle}% 如果需要移除整章的页眉

随着工业检测精度要求越来越高,对于近机类专业的学生来说,工程光学无处不在。所以对于非光学专业的我们,
掌握光学的基本知识已成为一门必不可少的技能之一。这正是我当初选择丁红昌老师开设的《工程光学》和熊家新老师开设的
《光电检测技术》的缘由之一,因为丁红昌老师教过我们《误差理论与数据处理》,所以我对丁老师的印象十分深刻,我比较喜欢听老师们在
课上讲解一些工程实践案例,这一般需要很多年经验的积累而形成的,而丁老师经常会在
课上讲解一下他科研时遇到的困难和解决方法的思路,这对我们以后从事科研或者工作启发甚大,感谢丁老师的经验分享,让我们受益良多。

在这次《工程光学》课程中,我学习了不少关于工程光学的知识,如光电编码器、OCT、光机系统设计、望远镜种类、光学频率梳技术、
镜片材料特性和著名镜头厂家、景深、视场等,开阔了自己的视野,而且从老师讲解的课题中,也了解到中国在哪些领域与国外存在差距,
哪些领域现比较热门,哪些地方还需要继续攻关。老师讲的知识面很广,打开了我的视野,而我现水平还有很大的差距,有些知识还需要课后慢慢查资料,仔细的研究一下才能弄懂。我从老师讲解引导、课后搜索文献和查找资料这过程中,学习到不同的思路和方法,为下学期毕设的调查和研究,经历了一个初步实践的过程,也认识到了光学应用无处不在,也坚定了我自学光学知识的决心。希望以后我能有机会和丁老师您讨论一下学术问题,交流工作经验,我一直希望能到老师您的实验室参观和实践一下,因为老师您有很多的器件供我们学习之用,可惜我大四之前不知道您的实验室
面向全校开发,错了很多学习的机会。

丁老师,我觉得您上次课后讲的提议相当的不错,向现有的企业申请一两天去做实验,因为企业的仪器设备一般都是
比较先进和好用的,最重要的是能用。我当初在大二主教做的电子技术实验,经常遇到很多器件损坏而无法正常使用的状况,十分懊恼,而且使同学们对实验失去信心;还有大三在机电实验楼做的自动控制实验,也是遇到校内设备经常不能正常使用的问题,为了更好的理解知识,只能通过软件来进行模拟仿真,但是模拟仿真很难考虑现实实验中存在的各种突发情况,学会在实践中利用器件进行排错也是一门必不可少的技能;最后一个问题是,校内的有的器件实在太旧了,不能每次说这个器件经典,因而从它学起,这没有问题,但是新颖的器件不代表就没有经典的结构存在。所以我觉得
老师您的提议相当的不错。希望以后的学弟和学妹能用上好的器件进行实践学习。

最后,我们十分感谢丁老师对《工程光学》课程的专心备课,让我们学习到不少的光学知识,收益匪浅,获益良多。



%\cleardoublepage[plain]% 让文档总是结束于偶数页,可根据需要设定页眉页脚样式,如 [noheaderstyle]
