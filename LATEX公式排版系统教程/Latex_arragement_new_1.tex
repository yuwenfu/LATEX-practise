\documentclass{book}

\usepackage{mathrsfs}
\usepackage{amsmath,amssymb}
\usepackage{amsthm}
\usepackage{mathtools,mathtext}
\usepackage{easybmat} 
		%编排分块矩阵需要用到水平虚线和垂直虚线,可调用块矩阵宏包 easybmat,它提供了一个BMAT块矩阵环境,该环境命令结构为:
		%\begin{BMAT}(格式矩阵){列格式}{行格式} \end{BMAT}
\begin{document}
\chapter{intro}
$$
	\left.
	\begin{array}{llll}
		 \text{常数} & y & = & c \\ 
		 \text{直线} & y & = & cx+d \\
		 \text{抛物线} & y & = & bx^2+cx+b
	\end{array}
	\right\} \text{多项式} %\leqno{a.0}
$$

$$
	\left.
	\begin{array}{llll}
	\text{常数} & y & = & c \\ 
	\text{直线} & y & = & cx+d \\
	\text{抛物线} & y & = & bx^2+cx+b
	\end{array}
	\right\} \text{多项式} \leqno(a.0)
$$

\begin{equation}
	\sum_{0 \leqslant i\atop 0<j<n} P(i,j)
\end{equation}

\begin{equation}
\sum_{\begin{subarray}{l}
0 \leqslant i \\ 0<j<n
\end{subarray}}P(i,j)
\end{equation}

$$\sqrt{2}<\sqrt[3]{3}$$
$$\sum\limits_{k=1}^n,\int\limits_a^b$$
$$\int_{a}^{b}$$

%数学重音符号
\[
\widehat{ab} + \widehat{cdef} = \widetilde{xyz}
\]
%上画线与下画线
$$	\overline{abc...def} $$
$$	\underline{abc...def} $$
$$	\overset{132}{\overbrace{abc...def}} $$
$$	\underbrace{abc...def}$$

%连加,连乘
\[  \sum_{i=1}, \sum\limits_{i=1}^n  \]
\[ \prod_{i=1}, \prod\limits_{i=1}^n \]

%在符号底部写符号:\underset
\[ \underset{0 \leq j \leq k-1}{argmin} \]
%在符号下部换行:\understack
$$ \sum_{\substack{0<i<n \\ 0<j<n}}A_{ij} $$

\chapter{intro1}
%矩阵
\section{matrix}
\begin{gather}
	\begin{matrix}
	1 & 0 \\ 
	0 & -1 
	\end{matrix}	
	\begin{pmatrix}
	1 & 0 \\
	 0 & -1 
	\end{pmatrix}	
	\begin{Bmatrix}
	1 & 0 \\
	0 & -1
	\end{Bmatrix}\\	
	\begin{bmatrix}
	1 & 0 \\
	0 & -1 
	\end{bmatrix}	
	\begin{Vmatrix}
	1 & 0 \\
	0 & -1 
	\end{Vmatrix}			
	\begin{vmatrix}
	1 & 0 \\
	0 & -1
	 \end{vmatrix}	
\end{gather}

\begin{equation}
	\mathbf{A}_{m,n} = 
	\begin{pmatrix}
	a_{11} & \dots & a_{1n} \\
	\vdots & \ddots & \dots \\
	a_{m1} & \dots & a_{mn} 
	\end{pmatrix}
\end{equation}

%矩阵中的虚线
\begin{equation}
	A = 
	\begin{bmatrix}
	a_{11} & a_{12} & \dots & a_{1n}\\
	a_{21} & \dots & \dots & \dots\\
	\hdotsfor{4}\\
	a_{m1} & a_{m2} & \dots & a_{mn} 
	\end{bmatrix}
\end{equation}

%单位矩阵
\begin{equation}
	E=
	\begin{bmatrix}
	1\\
	& 1 &  & \text{{\huge 0}}\\
	& & 1 \\
	& \text{{\huge 0}} & & 1\\
	& & & & 1
	\end{bmatrix}
\end{equation}

%矩阵方程
\begin{gather}
	\underbrace{
	\begin{bmatrix}
	y_{1} & 1 & 1 \\[20pt]
	\frac{1}{\sqrt{2}} & 1 & y_{2} \\[20pt]
	1 & 1 &  y_{3} 
	\end{bmatrix}
	}_{Y_(3)}
	\underbrace{
	\begin{bmatrix}
	V_1 \\[20pt]
	V_2 \\[20pt]
	V_3 
	\end{bmatrix}
	}_{V_(3)}=0
\end{gather}

%矩阵方程组
\begin{equation}
	\begin{cases}
		\begin{array}{*{3} {l@{+}} l @{=} l}
		a_{11}x_{1} & a_{12}x_{2} & \cdots & a_{1n}x_{n} & c_{1}\\
		a_{21}x_{1} & a_{22}x_{2} & \cdots & a_{2n}x_{n} & c_{2}\\
		\hdotsfor{5}\\
		a_{m1}x_{1} & a_{m2}x_{2} & \cdots & a_{mn}x_{n} & c_{m}\\		
		\end{array}
	\end{cases}
\end{equation}

\begin{equation}  
	\left\{
	\begin{array}{rrrl}
	a_{0}+a_{1}x_{0}+...+a_{n}x_{0}^{n}=y_{0} \\
	a_{0}+a_{1}x_{1}+...+a_{n}x_{1}^{n}=y_{1} \\
	\cdots\\
	a_{0}+a_{1}x_{n}+...+a_{n}x_{n}^{n}=y_{n}
	\end{array}
	\right.
\end{equation}


\[
	\begin{cases}
		u_{tt}(x,t)= b(t)\triangle u(x,t-4)   \\
		q(x,t)f[u(x,t-3)]+te^{-t}\sin^2x,    				\hspace{42pt}& t\neq t_k; \\
		u(x,t_k^+) - u(x,t_k^-) = c_k u(x,t_k),				 & k=1,2,3\ldots ;\\
		u_{t}(x,t_k^+) - u_{t}(x,t_k^-) =c_k u_{t}(x,t_k), 	&k=1,2,3\ldots
	\end{cases}
\]

%分段函数
\[ 
	q(x,t)= 
	\begin{cases}
	(t-k+1)x^2,\quad  	& t\in \big( k-1,k-\dfrac{1}{2} \big],\\ 
	(k-t)x^2, \quad 	& t\in \big( k-\dfrac{1}{2},k \big]
	\end{cases} 
\] 

%分块矩阵
%编排分块矩阵需要用到水平虚线和垂直虚线,可调用块矩阵宏包 easybmat,它提供了一个BMAT块矩阵环境,该环境命令结构为:
%\begin{BMAT}(格式矩阵){列格式}{行格式} \end{BMAT}
\begin{equation}  
	A= 
	\begin{bmatrix}
		\begin{BMAT}(@,30pt,20pt){cc.c}{cc.c}
		a_{11} & a_{12} & a_{13}\\
		a_{21} & a_{22} & a_{23}\\
		a_{31} & a_{32} & a_{33}
		\end{BMAT}	
	\end{bmatrix}
\end{equation}

%行内矩阵
$$
F=
\big(
	\begin{matrix}
		A & B \\
		C & D
	\end{matrix}
)
$$
$$
F=
\big(
	\begin{smallmatrix}
		A & B \\
		C & D
	\end{smallmatrix}
)
$$

\end{document}