\documentclass{myarticle}
\usepackage{graphics}
\usepackage{graphicx}
\usepackage{amsmath}
\usepackage{multirow} %用于表格合并行
\usepackage{array}
\usepackage{booktabs}
\usepackage{array}
\usepackage{colortbl}
\usepackage{bigstrut}



\begin{document}
%	%boldemph.sty
\ProvidesPackage{boldemph}[2018/12/22 v1.0 emphasis using bold font]
\newcommand\Emph[1]{\textbf{#1}}

\let\boldemph@oldemph\emph
\def\emph#1{\textbf{\boldemph@oldemph{#1}}}
	\Emph{ablggg}
	\emph{aafa你好}aa号
	
	\begin{quotation}
		\settext{emph=\textbf} %粗体
		An \emph{important} example.
		
		\settext{emph} %默认格式
		An \emph{important} example.
				
	\end{quotation}
% Table generated by Excel2LaTeX from sheet '光纤位移'
\begin{table}[!htbp]
	\flushleft
	\caption{Add caption}
	\begin{tabular}[scale=0.7]{|l|llllllllll|}
		\hline
		\rowcolor[rgb]{ .749,  .749,  .749} \multicolumn{11}{|c|}{\textcolor[rgb]{ 0,  .439,  .753}{光纤位移传感器输出电压与位移数据}} \bigstrut\\
		\hline
		\rowcolor[rgb]{ .749,  .749,  .749} \textcolor[rgb]{ 0,  .439,  .753}{位移X(mm)} & \cellcolor[rgb]{ 1,  1,  1}0 & \cellcolor[rgb]{ 1,  1,  1}0.1 & \cellcolor[rgb]{ 1,  1,  1}0.2 & \cellcolor[rgb]{ 1,  1,  1}0.3 & \cellcolor[rgb]{ 1,  1,  1}0.4 & \cellcolor[rgb]{ 1,  1,  1}0.5 & \cellcolor[rgb]{ 1,  1,  1}0.6 & \cellcolor[rgb]{ 1,  1,  1}0.7 & \cellcolor[rgb]{ 1,  1,  1}0.8 & \cellcolor[rgb]{ 1,  1,  1}0.9 \bigstrut[t]\\
		\rowcolor[rgb]{ .749,  .749,  .749} \textcolor[rgb]{ 0,  .439,  .753}{电压U(V)} & \cellcolor[rgb]{ 1,  1,  1}0 & \cellcolor[rgb]{ 1,  1,  1}0.013 & \cellcolor[rgb]{ 1,  1,  1}0.093 & \cellcolor[rgb]{ 1,  1,  1}0.176 & \cellcolor[rgb]{ 1,  1,  1}0.268 & \cellcolor[rgb]{ 1,  1,  1}0.35 & \cellcolor[rgb]{ 1,  1,  1}0.464 & \cellcolor[rgb]{ 1,  1,  1}0.612 & \cellcolor[rgb]{ 1,  1,  1}0.805 & \cellcolor[rgb]{ 1,  1,  1}0.913 \\
		\rowcolor[rgb]{ .749,  .749,  .749} \textcolor[rgb]{ 0,  .439,  .753}{U=1.0486X-0.1025} & \cellcolor[rgb]{ 1,  1,  1}-0.1025 & \cellcolor[rgb]{ 1,  1,  1}0.00236 & \cellcolor[rgb]{ 1,  1,  1}0.10722 & \cellcolor[rgb]{ 1,  1,  1}0.21208 & \cellcolor[rgb]{ 1,  1,  1}0.31694 & \cellcolor[rgb]{ 1,  1,  1}0.4218 & \cellcolor[rgb]{ 1,  1,  1}0.52666 & \cellcolor[rgb]{ 1,  1,  1}0.63152 & \cellcolor[rgb]{ 1,  1,  1}0.73638 & \cellcolor[rgb]{ 1,  1,  1}0.84124 \\
		\rowcolor[rgb]{ .749,  .749,  .749} \textcolor[rgb]{ 0,  .439,  .753}{非线性误差差值} & \cellcolor[rgb]{ 1,  1,  1}0.1025 & \cellcolor[rgb]{ 1,  1,  1}0.01064 & \cellcolor[rgb]{ 1,  1,  1}-0.01422 & \cellcolor[rgb]{ 1,  1,  1}-0.03608 & \cellcolor[rgb]{ 1,  1,  1}-0.04894 & \cellcolor[rgb]{ 1,  1,  1}-0.0718 & \cellcolor[rgb]{ 1,  1,  1}-0.06266 & \cellcolor[rgb]{ 1,  1,  1}-0.01952 & \cellcolor[rgb]{ 1,  1,  1}0.06862 & \cellcolor[rgb]{ 1,  1,  1}0.07176 \\
		\rowcolor[rgb]{ .749,  .749,  .749} \textcolor[rgb]{ 0,  .439,  .753}{非线性误差(\%)} & \cellcolor[rgb]{ 1,  1,  1}11.39  & \cellcolor[rgb]{ 1,  1,  1}-10.64  & \cellcolor[rgb]{ 1,  1,  1}7.11  & \cellcolor[rgb]{ 1,  1,  1}12.03  & \cellcolor[rgb]{ 1,  1,  1}12.24  & \cellcolor[rgb]{ 1,  1,  1}14.36  & \cellcolor[rgb]{ 1,  1,  1}10.44  & \cellcolor[rgb]{ 1,  1,  1}2.79  & \cellcolor[rgb]{ 1,  1,  1}-8.58  & \cellcolor[rgb]{ 1,  1,  1}-7.97  \bigstrut[b]\\
		\hline
	\end{tabular}%
	\label{tab:addlabel}%
\end{table}%


	
\end{document}