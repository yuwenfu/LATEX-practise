\documentclass{ltxdockit}
\usepackage{btxdockit}
\usepackage[utf8]{inputenc}
\usepackage[american]{babel}
\usepackage[strict]{csquotes}
\usepackage{tabularx}
\usepackage{longtable}
\usepackage{booktabs}
\usepackage{shortvrb}
\usepackage{pifont}

\rcsid{$Id: cleanthesis.tex,v 0.3 2013/05/12 23:50:00 derric stable $}

\newcommand*{\cleanthesis}{\emph{Clean Thesis}\xspace}
\newcommand*{\cthesishome}{http://cleanthesis.der-ric.de/}
%\newcommand*{\cthesisctan}{http://www.ctan.org/tex-archive/macros/latex/contrib/../}

\titlepage{%
  title={The \sty{cleanthesis} Package},
  subtitle={A LaTeX Style for Thesis Documents},
  url={\cthesishome},
  author={Ricardo Langner},
  email={info@cleanthesis.der-ric.de},
  revision={\rcsrevision},
  date={\rcstoday}}

\hypersetup{%
  pdftitle={The \cleanthesis Package},
  pdfsubject={A LaTeX Style for Thesis Documents},
  pdfauthor={Ricardo Langner},
  pdfkeywords={tex, latex, thesis, style}}


%\setcounter{secnumdepth}{4}

\begin{document}

\printtitlepage
\tableofcontents
\listoftables

\section{Introduction}
\label{sec:intro}

\subsection[About]{About \sty{cleanthesis}}
\label{sec:intro:about}

\subsection{License}
\label{sec:intro:license}

Copyright \textcopyright\ 2011--2013 Ricardo Langner.
\cleanthesis is free software: you can redistribute it and/or modify it under the terms of the GNU General Public License as published by the Free Software Foundation, either version 3 of the License, or (at your option) any later version.

\cleanthesis is distributed in the hope that it will be useful, but WITHOUT ANY WARRANTY; without even the implied warranty of MERCHANTABILITY or FITNESS FOR A PARTICULAR PURPOSE.
See the GNU General Public License for more details.

You should have received a copy of the GNU General Public License along with this program.
If not, see \url{http://www.gnu.org/licenses/}.

\subsection{Feedback}
\label{sec:intro:feedback}

\subsection{Acknowledgments}
\label{sec:intro:ack}

I would like to thank the following people for using the \cleanthesis style and giving feedback to me, e.g., features, bugs.

\begin{itemize}
\item \textbf{Anton Augsburg} in his project thesis \\ \url{http://antonaugsburg.de/} (in German only)
\item \textbf{Mathias Frisch} in his dissertation (PhD) \\ \url{http://wwwpub.zih.tu-dresden.de/~frisch/}
\item \textbf{Sebastian Kleinau} in his bachelor thesis \\ \url{http://www.sk-downloading.de/} (in German only)
\end{itemize}


\subsection{Prerequisites}
\label{sec:intro:pre}

The follwing section gives an overview of all resources required by this package.

\subsubsection{Requirements}
\label{sec:intro:req}

\section{User Guide}
\label{sec:userguide}

\subsection{Package Options}
\label{sec:userguide:pkgopt}

All package options are given in \keyval notation.
The value \texttt{true} can be omitted for all boolean keys, \eg \opt{sansserif} without a value is equivalent to \kvopt{sansserif}{true}.

All of the following options must be used as \sty{cthesis} is loaded, \ie in the optional argument to \cmd{usepackage}.

\begin{optionlist}

\boolitem[false]{sansserif}

Sets whether to use a sans serif font or not.

\boolitem[false]{hangfigurecaption}

\boolitem[true]{hangsection}

\boolitem[true]{hangsubsection}

\optitem[endash]{figuresep}{\opt{none},\opt{colon},\opt{period},\opt{space},\opt{quad},\opt{endash}}

This option can be used to define a different label separator for cations of figures. The following value are allowed:

\begin{valuelist}
\item[none] Inserts no character in between.
\item[colon] Inserts a colon (\textbf{:}) in between.
\item[period] Inserts a period (\textbf{.}) in between.
\item[space] Inserts a single space character in between.
\item[quad] Inserts a \cmd{\\quad} in between.
\item[endash] Inserts an en dash (\textbf{--}) in between.
\end{valuelist}

\optitem[full]{colorize}{\opt{full},\opt{reduced},\opt{bw}}

\optitem[bluemagenta]{colortheme}{\opt{bluemagenta},\opt{bluegreen}}

\optitem[bibtex]{bibsys}{\opt{biber},\opt{bibtex}}

Sets whether to use \texttt{biber} or \texttt{bibtex} as citation management tool (engine).
The default (still) is \texttt{bibtex}.

"\texttt{Biber} [is] a BibTeX replacement for users of BibLaTeX", see <http://biblatex-biber.sourceforge.net/>.

\end{optionlist}

\end{document}
