\documentclass{ctexart}

\usepackage{geometry}%更改页边距所需  
\usepackage{graphicx}%插入图片所需  
\usepackage{booktabs}%表格线条加粗所需  

\usepackage{cite}		%用于配置引用的格式
						%cite引文宏包已被natbib更强大的引文宏包替代
%\usepackage{natbib}	%功能更强大的引文宏包,具体查文档说明
							
%\usepackage{hyperref}	%给引用生成超链接

\bibliographystyle{unsrt}	%声明参考文献的格式
%plain:按作者、日期、标题的字母顺序排序
%unsrt:按正文引用文献的顺序排序
%alpha:作者姓氏前三个字母+出版年份后两位
%abbrv:缩写作者名和月份名,其余同plain

\title{参考文献环境的使用}
\author{YW Fu}
\date{2018年11月11日}
%\geometry{left=2.54cm,right=2.54cm,top=1.0cm,bottom=1.0cm}%更改页边距  

\begin{document}%开始正文  
	\maketitle 
	\begin{abstract}	%摘要
		这是写摘要。
	\end{abstract}	
	\section{Intro}
	参考文献可以这样引用
	
	%\nocite{*}	%在列表显示并不直接引用的文献
	
	这是正文。这是正文\cite{.}\cite{Das.2018}。
	
	\newpage
	\bibliography{./Bibliography/123}
	
	
	%\bibliography{./math} %右键打开math.bib文件
	%\bibliography{./biobliography/math,./biobliography/CNKI_MEMS,WOS_MEMS} %右键打开math.bib文件
	%\bibliography{./biobliography/WOS_MEMS,./biobliography/CNKI}

			
%	\begin{thebibliography}{}
%		\bibitem{MEMSa}
%		付煜文,yuwenfu
%		{\em {Introduction to MEMS}}.
%		北京:清华大学出版社,2020.
%		\bibitem{MEMSb}
%		付煜文,yuwenfu
%		{\em {Introduction to MEMS}}.
%		北京:清华大学出版社,2021.
%		\bibitem{MEMSc}
%		付煜文,yuwenfu
%		{\em {Introduction to MEMS}}.
%		北京:清华大学出版社,2022.
		 
%		\bibitem{ref1.1}%加标签  
%		Li, J.; Bioucas-Dias,J.M.; Plaza, A. Spectral–spatial hyperspectral image segmentation usingsubspace multinomial logistic regression and Markov random fields. {\em IEEETrans. Geosci. Remote Sens.} {\bf 2012}, {\em 50}, 809-823.  
%	\end{thebibliography}%结束编辑参考文献  


\end{document}%结束正文</code>  
