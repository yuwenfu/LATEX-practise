\documentclass[UTF8,fontset = windows,zihao=-4,scheme=chinese,space=auto,linespread=1.25,twocolumn]{ctexart}

\usepackage{graphicx}
\usepackage[]{gbt7714}
\usepackage{lipsum}

\usepackage{titletoc} %改变目录格式

\ctexset{today=big,section/numbering= false} %big/small/old;numbering控制是否对不带星号的标题命令进行编号


\ctexset{contentsname = {目 录}, bibname = {文献列表}, abstractname = {摘要}, bibname = {参考文献}}

\title{\heiti CTeX 宏包套件使用介绍}
\author{\kaishu 付煜文}

\date{\today}


\begin{document}
		\maketitle
		
		\let\saved\thepage  %把thepage保存到saved宏里面
		\let\thepage\relax	%relax表示什么都不操作,将其放在thepage里面
		
		\begin{abstract}
			\lipsum[1]
		\end{abstract}
		
		\newpage
		\tableofcontents	%输出目录
		
		\newpage
		\let\thepage\saved %把saved里面的东西还原到thepage
		\setcounter{page}{1}	%设置页码
		
		\section{CTeX 简介}
		\subsection{字体简介}
		\songti 这是\cite{.2004b}宋体字体 songti。this is the songti code. \\
		
		\newpage
		\section{这是正文文献引用示例}
		这是一遍正文。\cite{RN169}这是一遍正。\cite{.2005c}。这是一遍正文。这是一遍正文。这是一遍正文。这是一遍正文。这是一遍正文。这是一遍正文。这是一遍正文。这是一遍正文。这是一遍正文。这是一遍正文。这是一遍正文。这是一遍正文。这是一遍正文。\lipsum[3-5]

		
		\clearpage	 %类似于新起一页
		\addcontentsline{toc}{section}{参考文献}  %将参考文献加入目录
		\bibliography{bibliography/20181115}
\end{document}
