\chapter{光机结构设计}
这一章,我将介绍激光通信系统主动端的二自由度(俯仰轴和方位轴)回转平台的工作原理、光学基台是如何安装在俯仰回转平台上的、光学基台的组成和MEMS逆向光调制器的机械安装装置。
\section{主动端二自由度回转平台组成}

\addimg{1}{2DOFmechanics.jpg}{二自由度回转平台}



主动端的回转平台共有两个自由度,一个是垂直于水平面的方位轴,另一个是平行于水平面的俯仰轴,如图~\ref{2DOFmechanics.jpg}所示。

\addimg{1}{2DOF-explore.jpg}{二自由度回转平台爆炸图}

从回转平台的爆炸图可以看出,如图~\ref{2DOF-explore.jpg},方位控制和俯仰控制都采用直流力矩电机进行驱动,为了保证系统的抗干扰性和转向的准确,两个自由度上都采用了光栅进行角度测量,实时进行角度误差的校正,保证旋转的精度。

\subsection{方位转台设计}
方位转台设计采用了两个角接触轴承,将其安装在轴套里面,对U型架进行支撑。从图~\ref{2DOF-buttom.jpg}可以看出由读数头和光栅组成的角度检测装置,对方位角进行实时的测量。
\addimg{1}{2DOF-buttom.jpg}{方位轴角度检测光栅}

图~\ref{2DOF-yaw.jpg}展示的是去掉U型架之后,方位轴总装的顶视图,图中有动静两个限位块对方位角进行限制,所以说主动端并不能360°的旋转,有一定的范围。
\addimg{1}{2DOF-yaw.jpg}{方位轴直流力矩驱动电机}

\subsection{俯仰转台设计}
俯仰方向也是采用直流力矩驱动电机作为动力源,也是两端分别采用两个角接触轴承进行支撑,将光学平台和前后罩组成的负载的重量传递给U型架,如图~\ref{2DOF-pitch-motor.jpg}。
\addimg{1}{2DOF-pitch-motor.jpg}{俯仰轴直流力矩驱动电机}

从图~\ref{2DOF-grating.jpg}可以看出俯仰轴也采用了光栅进行角度检测和都两个动静限位块组成的行程控制机构。
\addimg{1}{2DOF-grating.jpg}{俯仰轴角度检测光栅}


\subsection{光学平台设计}
光学平台作为逆向光通信系统的核心部分,对机械安装精度非常高,而且组成部分十分复杂,如图~\ref{optics-front.jpg}和~\ref{optics-back.jpg}。

\addimg{0.5}{optics-front.jpg}{光学平台前视图}
\addimg{0.5}{optics-back.jpg}{光学平台后视图}

从图~\ref{optics-dig.jpg}可以看出,已调制信息的激光,经卡塞格林系统接收后,先经过光束整形光组,对光束进行压缩和近准直,减少能量的损失,最后经过棱镜转向后,光束到达光电探测器,完成信息的接收。
\addimg{0.6}{optics-dig.jpg}{光学平台探测器安装位置}

\addimg{1}{optics-explorer.jpg}{光学平台的爆炸视图}
从图~\ref{optics-explorer.jpg}可以看出,主动端的激光通信系统由询问激光发射器、信标光发射器、信标光接收器、靶标相机、卡塞格林系统、光束整形和转向光组和探测器组成。




\section{MEMS逆向光调制器(角反棱镜式)机械结构设计}
MEMS调制器外形特别小。实际应用时除了有MEMS调制器外,还需要电源模块和电路控制模块等,所以需要设计额外的机械结构对调制器、电源模块和相应的电路模块进行固定,方便后续实际工程测试的时候能固定在轻小型平台上,所以设计了如图~\ref{MRR-DIG.jpg}的机械结构。

\addimg{0.5}{MRR-DIG.jpg}{MEMS调制器机械安装结构}



\addimg{1}{MRR-explorer.jpg}{MEMS调制器爆炸图}

如图~\ref{MRR-explorer.jpg}为MEMS调制器机械固定结构的爆炸图,分别设计了MRR安装前盖和后盖的模槽对MEMS调制器和电池组进行固定,然后电路板固定采用螺钉直接固定在MRR安装后盖上,然后在电池组上放置支撑块,对MRR外壳进行支撑,最后用外壳固定架固定即可。




