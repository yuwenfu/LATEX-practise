\chapter{激光单元}
\section{激光器概述}
对于近地激光通信光端机,若采用直接探测模式,则要有高效、多吉比特每秒带宽调制能力、中等平均功率发射水平和高质量空间光束的激光发射能力,而某些光纤放大器和半导体二极管激光器是最佳选择。目前,已有的非定制的激光发射器可达到近地面激光通信链路对输出功率和光束质量的要求。除了在抗辐射性和真空中的环境下应用尚未验证之外,而 Telcorida的商用激光器满足几乎所有的机载和星载任务的指标要求。需要改善的是提高整体工作效率(激光驱动器和抗热性)以及执行10\~{}15年持久任务的寿命。
激光通信中波长主要集中在810m、1064m和1550m。其中最常用的波长为1550nm,不仅是1550mm激光的商品化率高,而且此波段的大气衰减和对人眼的危害也比800nm、1064mm波长要小得多,安全性高。考虑到链路预算和波长之间的关系,波长的大小对其影响更明显。总之,激光的选择依据主要考虑器件可用性、足够大功率并易于调制。

目前,利用光放大器如掺铒光纤放大器的半导体激光器可将输出数据传输速率超过10Gb/s,输出功率从1mW提高到数千毫瓦。振荡器以多吉比特每秒的速率输出调制光信号之后经光放大且保持其较好的光束质量特性。激光发射单元的设计包括波长、输出功率、增益、增益系数、增益波动、噪声系数和带宽等参数。图~\ref{guangxianfangdaqi.pdf}是一个多级振荡一放大器框图

\addimg{0.8}{guangxianfangdaqi.pdf}{几十dBm调制激光功率的多级放大器,输出功率根据放大器级数扩展}

\section{激光束瞄准和稳定}
对于运动平台间的自由空间激光通信,激光束指向和对准是最具挑战性的技术之一。激光通信指向在一般情况下可以分解成光视轴稳定和提供正确的接收器位置。后者由接收端信标光完成,前者通过使用高带宽的控制回路来正确感应平台抖动。近地激光通信系统接收端的强信标光同时可完成以上两功能。
\section{激光单元的实验设备理论计算和选型}

\subsection{理论计算}
实验采用的激光器是科乃特公司的1.55um系列光纤激光器。

\addimg{1}{VFLS-1550-B.jpg}{VFLS-1550-B光纤激光器}

由光纤激光器输出的激光束散角太大,导致传输距离有限,且能量损失严重,所以必须在光纤输出端加光纤准直器,光纤准直器通过光纤尾纤与透镜精确定位而成,把光纤内的传输光变为准直光耦合到单模光纤内。

查询科乃特公司(connet laser)的1.55um系列连续光纤激光器的技术文档可知,如图~\ref{laser-guangxian.pdf}所示,单模光纤是采用康宁公司最常见的SMF28系列的,连接器为APC结构,设计和选型时需要注意。

\addimg{0.9}{laser-guangxian.pdf}{1550nm连续光纤激光器的技术文档}

查询Corning公司SMF-28e+光纤的技术手册可知,如图~\ref{Corning-SMF28E.pdf}。

\addimg{1}{Corning-SMF28E.pdf}{Corning-SMF28E技术文档}

具体的光学指标,如模场直径([MFD])、色散和数值孔径(NA)如图~\ref{MFD.png}、图~\ref{SMF28-sesan.pdf}和~\ref{SMF28-NA.pdf}所示,这些参数主要用于设计光纤准直透镜时会用到。


\addimg{0.8}{MFD.png}{Corning-SMF28e+的模场直径}
\addimg{0.8}{SMF28-sesan.png}{Corning-SMF28e+色散}
\addimg{0.8}{SMF28-NA.pdf}{Corning-SMF28e+数值孔径NA}
\addimg{0.8}{Divergence.jpg}{光纤激光准直器的光束直径}
根据输出光束直径(7mm)估算准直器的焦距,如图\ref{Divergence.jpg},所用激光波长为1550nm,公式如下
$$ f \approx \frac { \pi d [ M F D ]} {  4 \lambda  }  \approx \frac { \pi \times 7 \times 10.4 \times 10^{-6}} {  4 \times 1.55 \times 10^{-6}  } \approx 36.89 mm $$
根据单模光纤估算其发散角(全角),焦距取37mm,公式如下
$$ \theta \approx \left( \frac { [ M F D ] } { f } \right) \left( \frac { 180 } { \pi } \right) =  \left( \frac { 10.4 \times 10^{-6}} { 37 \times 10^{-3} } \right) \left( \frac { 180 } { \pi } \right) =  0.0161 ^{\circ} $$

维持准直,激光束束腰距离透镜最远的距离,根据计算公式,可知
$$ z _ { \max } = f + \frac { 2 f ^ { 2 } \lambda } { \pi [ M F D ] ^ { 2 } } =  37\times 10^{-3} + \frac { 2 \times\left(37\times 10^{-3}\right)^ { 2 } \times 1.55\times 10^{-6} } { \pi \times\left({10.4\times 10^{-6}} \right)^ { 2 } } \approx 12.527 m  $$

\subsection{准直透镜设备选型}
根据上述的计算,在THORLBAS公司的空气隙双合透镜准直器产品选型即可,如表\ref{fiber-xuanxi.png}和图\ref{F810FC.jpg}。
\addimg{1}{F810FC.jpg}{带准直器的光纤激光器}
%[1550nm连续光纤激光器技术参数](http://connet-laser.com.cn/ProDetail.aspx?pid=95&posi=3)
\addimgTbl{1}{fiber-xuanxi.png}{选型}

\subsection{准直透镜ZEMAX仿真}
根据计算的参数选型,将模型导入ZEMAX仿真。

准直透镜参数如图~\ref{F810APC-1550-01.png}所示。

\addimg{1}{F810APC-1550-01.png}{准直透镜参数}
准直透镜布局图如图\ref{Fiber-Collimation.pdf}所示。

\addimgRt{0.6}{Fiber-Collimation.pdf}{准直透镜ZEMAX布局图}{90}
准直透镜圈内能量图如图\ref{F810APC-1550-ENY.pdf}所示。

\addimgRt{0.6}{F810APC-1550-ENY.pdf}{准直透镜圈内能量图}{90}
准直透镜的光线扇形图如图\ref{F810APC-1550-FAN.pdf}所示。

\addimgRt{0.6}{F810APC-1550-FAN.pdf}{准直透镜的光线扇形图}{90}
准直透镜的点列图如图\ref{F810APC-1550-SPOT.pdf}所示。

\addimgRt{0.6}{F810APC-1550-SPOT.pdf}{准直透镜的点列图}{90}




\addimg{0.8}{Fiber.jpg}{带准直器的光纤激光器安装图}

\addimg{1}{zhunzhitoujinganzhuang.png}{激光准直器在光学基台的安装}

\subsection{激光准直器在光学基台的安装}
图~\ref{zhunzhitoujinganzhuang.png}显示的是激光准直器接上光纤激光发射器后,是如何安装在光学基台上的。


