\chapter{引言}\label{chap:introduction}

\section{研究背景}
{\href{https://zh.wikipedia.org/wiki/\%E8\%AE\%A1\%E7\%AE\%97\%E6\%9C\%BA\%E8\%A7\%86\%E8\%A7\%89}{\textbf{计算机视觉}}\footnote{计算机视觉:\url{https://zh.wikipedia.org/wiki/\%E8\%AE\%A1\%E7\%AE\%97\%E6\%9C\%BA\%E8\%A7\%86\%E8\%A7\%89}}
在国民经济,科学研究及国防建设等领域都有着广泛的应用。


\chapter{概述}

\section{视觉检测技术的相关概念}
视觉检测技术\citep{yong.2013}是一门面向特定视觉任务,建立在计算机视觉和图像处理基础之上,对目标对象进行定性检测和定量检测的一门新兴检测技术

\subsection{计算机视觉}

计算机视觉\citep{A.2017,zhang.2017}是研究人类视觉的计算模型,利用计算机对描述景物的图像数据进行处理,以实现类似人的视觉感知功能,对客观世界的三维场景进行感知、识别和初理解,是计算机科学和自然科学的重要组成部分。计算机视觉的研究方法主要有:

一、仿生学的方法~ 参照人类视觉系统的机构原理,建立相应的处理模型完成类似的功能和工作。

二、工程的方法~从分析人类视觉过程的功能入手,并不刻意模拟,人类视觉内部结构。



\subsection{狭义机器视觉}

狭义机器视觉的概念是指工业视觉检测,与普通计算机视觉、模式识别、数字图像处理有明显区别,是计算机视觉最重要的应用之一。目前,最权威的机器视觉的定义是美国制造业工程师协会和机器人工业协会给出的:
\begin{myqt}
	“机器视觉是利用非接触式的光学传感器自动采集实景物图像并进行处理,以获得所需的信息,并控制机器和生产过程的装置。”
\end{myqt}


