\chapter{引言}\label{chap:introduction}

\section{本课题研究目的和意义}
我正在研究基于MEMS的逆向光通信系统设计的课题。因为我尝试找出
\begin{itemize}
	\item 信息如何通过光介质来进行传输?为什么要使用光进行通信?
	\item 何为逆向光通信?它与现今的自由光通信和光纤通信的区别何在?
	\item MEMS器件是如何进行光信息的调制?其国内外发展如何?局限性有哪些?
	\item 常用的逆向调制器有哪些?国内外分别进行哪些相关的研究?研究进展如何?
	\item 为什么需要研究逆向光通信?它可以给现今带来哪些突破?
	\item 新兴的光通信方式有哪些?
\end{itemize}

为了深入了解基于MEMS逆光调制器的逆向光通信的原理,我将通过具体的电路设计,主动端光机结构的设计和具体的光路图来介绍语音信号是怎么通过单片机进行发送和接收,具体讲述单片机如何通过电平匹配电路将语音信息的数据加载至MEMS逆光调制器上,激光如何经过主动端的光学系统进行发送,并通过光电探测器接收经MEMS逆光调制器反射回来的已加载信息的激光,通过单片机进行解码还原出原有的语音信息,最后进行前后数据的对比,从而评估基于MEMS器件逆向光通信的可行性。

1.1.1\ 信息如何通过光介质来进行传输?为什么要使用光进行通信?

由于光的波粒二象性,光具有和电磁波一样的特性,如幅度、频率和相位,通过合理地设置这些参数就可以将信息有效地加载到光上,进行信息的传输,进而还发展出了抗干扰性能更好的相干光通信\citep{Honghui.2013},除此之外,光在通过某些特殊的材料后,其偏振态会发生改变(线偏振和圆偏振),进而可以设计出相应的调制器如图~\ref{FLC-SWITCH.pdf}。
\addimg{0.8}{FLC-SWITCH.pdf}{铁电液晶调制器}

利用光进行通信有很长的历史可以追溯,但是近代应用最广、发展最快的并不是光通信,在梅曼研制出激光器到康宁玻璃公司光纤商业化来看,至少是这样的。而最早的信息传输是通过电磁波进行信息传输,容易受磁场干扰,且带宽低,传输距离近,但是其优点是电磁波可以穿透物体传播,而光通信则不能。

现在由于大气信道对光信号严重的衰减效应没有得到有效的解决,大部分研究都集中在光纤通信器件的研制中,对于自由空间激光通信的研究发展比较缓慢。
\addimg{1}{THREE-FSO-PATTERN.pdf}{
	自由光通信系统 a 传统的光通信系统, b 基于逆向光调制器光通信系统 (半双工模式, HDX mode), c 基于逆向光调制器光通信系统 (全双工模式, FDX mode)}
传统自由光通信\citep{白帅.2015, 姜会林.2015}与逆向光通信区别如图~\ref{THREE-FSO-PATTERN.pdf}。

1.1.2\ {MEMS逆光调制器逆向光通信(MRR)原理,与现今的自由光通信(FSO)和光纤通信的区别。


逆向光通信(MRR)就是主动端使用传统光通信的激光系统(体积比较大)向被动端(即MEMS逆光调制器一端,没有ATP装置,只有一个MEMS逆光调制器件)发送一端激光束,MEMS逆光调制器件通过加载信息到激光束,然后将激光原路返回至主动端的技术,这样大大减少了被动端的体积和重量,能很方便的安装在各种小型平台上,而且MEMS逆光调制器需要的驱动电压比较低,具有低功耗的特点。

传统的激光通信两端的设备需要很高的对准精度,而且对视场角要求很高,通常在1°以内,导致光学装置十分复杂。而基于MEMS的逆向光调制器就没有这个限制,不但体积小,而且对视场角的要求比较宽裕,能达到30°或者更高,所以不需要复杂的光学对准和定位系统,有效解决了自由光通信的应用局限性。

同时MRR还具备光纤通信所不具备的机动性和灵活性,能在通信设备遭到破坏的时候迅速恢复,不需要花费进行光纤通信所需的大量铺设光纤的时间,但是MRR受大气影响较大,信息传输速率受到限制,没有光纤通信速度快和稳定。





\section{常用的逆向调制器类型和国内外文献发展现状}
逆光通信调制器\citep{李展.2007, 孙华燕.2013, 邱灏.2015b}从传统的声光(图~\ref{AOM-MRR.pdf})、电光、液晶等角立体棱镜调制器(图~\ref{corner-mrr.pdf})发展到如今的猫眼结构\citep{李展., 魏宾.b,任建迎.2017}调制器,MEMS(图~\ref{MEMS-MRR.pdf}) MRR、MQW逆向光调制器、电润湿透镜逆光调制器(图~\ref{Electrowetting-lenslet-MRR.pdf}),逆向调制速率从传输几kbps到如今的几Gbps的速率,从最初的半双工通信到如今的全双工通信,从最初的单点对单点的通信到如今的组网技术。
\addimg{0.6}{AOM-MRR.pdf}{声光调制器}
\addimg{1}{MEMS-MRR.pdf}{MEMS逆向光调制器原理图}
\addimg{1}{Electrowetting-lenslet-MRR.pdf}{电润湿透镜逆光调制器}
\addimg{0.6}{corner-mrr.pdf}{角反棱镜式逆向光调制器}

国外对MRR的研究普遍比国内研究得早,特别是在MQW MRR器件中,美国海军实验室(NRL)对MQW 器件的研究尤为深入,不断研制出实验模型,搭载到不同的移动平台上进行实验,积累了很多的经验。国内主要对MRR进行理论分析为主和提出了很多不同的调制方案以实现全双工通信,加快其商业化进程。目前国内研究MRR器件的高校主要集中在长春理工大学、西北工业大学等。国内外很多的高校、研究机构和公司都发布各自的理论分析和专利。以下整理成表格列出\citep{曾智龙.2017},如表~\ref{MRR-ABROAD.pdf}和表~\ref{MRR-INNER.pdf}。

\addimgTbl{1}{MRR-ABROAD.pdf}{国内逆向调制空间激光通信技术研究}
\addimgTbl{1}{MRR-INNER.pdf}{国外逆向调制空间激光通信技术研究}

从表~\ref{MRR-ABROAD.pdf}和表~\ref{MRR-INNER.pdf}对比可以看出,国内目前在逆光调制器产品研发的性能与国外还有不少的差距,主要基于理论验证,实物的调制速率不算高,离具体实际应用还有很长的路要走;而国外目前集中于量子阱(MQW)调制器的研发,研究机构为军事机构,研究成果带有保密性质,根据其最近实际工程调试的成果,可以看出MQW调制器传输数据的速率最快,技术比较成熟,积累了不少现场调试的经验。


\section{ 逆光调制器的发展趋势}
MEMS调制器件\citep{胡放荣.2010}主要依靠器件内部的机械运动或材料结构的变化来实现对光信息的调制,有学者研究表明MEMS器件机理决定了其最高通信速率不会超过几Gbps,所以现在国外很多学者都在研究基于量子阱的逆光调制器(MQW),还有学者通过整合猫眼镜头,形成了猫眼结构的量子阱调制器如图~\ref{CAT-MQW-ARRY.pdf},既满足大视场要求又有很高的调制速率。
\addimg{0.8}{CAT-MQW-ARRY}{带猫眼结构的量子阱调制器阵列}


逆向光通信在很多中小平台的得到了广泛的应用,如无人机通信、海上舰船通信、大气监测、海底通信、交会对接中对目标平台的姿态监控和目标识别等多个领域。

由于具有传统激光通信的抗干扰性、宽带宽等优点,还被应用于电磁环境复杂的通信链路中。当调制器采用阵列结构时,可以打破传统 FSO 链路只能点对点的工作模式,实现 FSO 链路组网\citep{.2016, 张雅琳.2016, JiangHuilin.2012},对天基信息实现网络化传输具有重要意义。

逆光调制系统由于有传统FSO没有的一点对多点的能力,还可以用于光学标签的识别。在医学应用领域,MRR还可以进行人体透皮无线光通信,如起搏器、智能假肢、大脑接口的神经信号处理器,充当人工眼睛和用于收集人体产生的信号的摄像头\citep{Majumdar.2015}。逆向调制激光通信系统,结构简单, 视场角大,对激光瞄准系统精度要求降低,节省经费,所以研究MRR FSO技术对国家发展有着重要的战略意义。


除了器件发展外,光通信形式也发生了改变。现在国内外很多机构和高校早已开始进行量子光通信的研究。量子光通信是以光子作为数据载体,由光量子态来携带信息,量子信息论表明量子光通信容量比经典光通信提高很多个数量级,安全系数很高,不存在被破译的可能\citep{yao.2011}。

光通信由于不能透光传输,光信息载体只能在真空、光纤或其他透光的物质内传播。中微子通信很有潜力成为新的通信方式,中微子质量很小,可以穿透任意物质,而且没有能量损耗,在空间应用中,可以不用中继卫星就可以实现与月球背面的端机进行通信,是未来很有发展可能的通信方式。



%\begin{myqt}

%\end{myqt}


