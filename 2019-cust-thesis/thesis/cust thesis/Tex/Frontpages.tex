%---------------------------------------------------------------------------%
%->> 封面信息及生成
%---------------------------------------------------------------------------%
%-
%-> 中文封面信息
%-
\confidential{}% 密级:只有涉密论文才填写
\schoollogo{scale=0.8}{cust_logo}% 校徽
\title{基于MEMS的逆向光通信系统设计\\Design of reverse optical communication system based on MEMS}% 论文中文题目
\author{付煜文}% 论文作者
\advisor{孟立新~长春理工大学}% 指导教师:姓名~专业技术职务~工作单位
\advisorsec{}% 指导老师附加信息 或 第二指导老师信息
\degree{学士}% 学位:学士、硕士、博士
\degreetype{工学}% 学位类别:理学、工学、工程、医学等
\studentnumber{150321128} %学号
\major{机械电子}% 二级学科专业名称
\institute{机电工程学院}% 院系名称
\chinesedate{二〇一九年六月}% 毕业日期:夏季为6月、冬季为12月
%-
%-> 英文封面信息
%-
\englishtitle{Design of reverse optical communication system based on MEMS }% 论文英文题目
\englishauthor{Fu Yuwen}% 论文作者
\englishadvisor{Supervisor: Professor Meng Lixin}% 指导教师
\englishdegree{Bachelor}% 学位:Bachelor, Master, Doctor。封面格式将根据英文学位名称自动切换,请确保拼写准确无误
\englishdegreetype{Engineering}% 学位类别:Philosophy, Natural Science, Engineering, Economics, Agriculture 等
\englishthesistype{thesis}% 论文类型: thesis, dissertation
\englishmajor{Mechatronic}% 二级学科专业名称
\englishinstitute{Institute of Mechanics, Changchun University of Science and Technology}% 院系名称
\englishdate{June, 2019}% 毕业日期:夏季为June、冬季为December
%-
%-> 生成封面
%-
\maketitle% 生成中文封面
\makeenglishtitle% 生成英文封面
%-
%-> 作者声明
%-
\makedeclaration% 生成声明页
%---------------------------------------------------------------------------%
%-> 中文摘要
%---------------------------------------------------------------------------%
\chapter[摘要]{摘\quad 要}\chaptermark{摘\quad 要}% 摘要标题
\setcounter{page}{1}% 开始页码
\pagenumbering{Roman}% 页码符号
%---------------------------------------------------------------------------%
本文针对轻小型平台安全保密、紧凑的要求,使用基于MEMS逆向光调制器件,开展了逆向光通信系统原理的研究。涉及到逆向光通信系统的光机结构是如何设计、理论计算和选型,最后涉及到软件的仿真、电路的设计、光学系统设计和光机结构的设计和建模。


具体工作如下:

1.设计主动端的RC卡塞格林光学系统,理论计算得出参数,最后通过ZEMAX优化,并确定遮光结构参数,SolidWorks设计固定光学系统的机械结构和系统的遮光结构。

2.通过SolidWorks设计主动端的两自由度回转平台,设计相应的回转轴系,对直流力矩电机进行选型,并设计固定的装置。

3.根据逆光调制器应用的场景和实际要求,进行选型和机械固定结构的设计。

4.根据所使用的的光纤激光器,理论计算确定激光光纤准直器的相关参数,然后依据参数进行产品的选型,获取光学参数后,进行ZEMAX仿真评估,看是否满足使用要求。

5.电路设计。根据要求需要设计一个语音采集电路,使用STM32单片机进行控制,对语音数据进行存储和操作;设计STM32单片机(CMOS电平)与MEMS逆向光调制器(TTL电平)的电平匹配电路,并用Multisim进行电路仿真,确定可行性和可靠性;通过STM32单片机对探测器接收的数据进行解码,解析出其中包含的语音数据;通过Altium Designer进行电路图绘制。

6.结合SolidWorks和CAD,进行3D建模,出工程图纸,包括主动端装配图,整个逆向调制系统的光路图和程序流程图。



%---------------------------------------------------------------------------%
\keywords{MEMS调制器;逆向光调制;ZEMAX;Multisum;激光通信;RC卡塞格林系统}% 中文关键词
%---------------------------------------------------------------------------%
%-> 英文摘要
%---------------------------------------------------------------------------%
\chapter*{Abstract}\chaptermark{Abstract}% 摘要标题
%---------------------------------------------------------------------------%
Based on light small compact platform security and requirements, this paper used modulating retro-reflector(MRR) based on MEMS, carried out the study of reverse optical communication system.Concerning the design, theoretical calculation and selection of the  opto-mechanical system structure of the reverse optical communication system. Finally came to the software simulation, circuit design,  opto-Mechanical system design and modeling.

The detailed works as follow:

1.The RC Cassegrain  optical system of the active end was designed and the parameters was calculated theoretically.Finally, the shading structure parameters was determined through ZEMAX optimization. SolidWorks was used to design the mechanical structure of the fixed optical system and the shading structure of the system.

2. Designed the 2-DOF rotary platform of the active end with SolidWorks, seted the corresponding rotation shaft system, selected the DC torque motor, and designed the fixed device.

3. According to the application scenarios and actual requirements of the backlight modulator, determined the mechanical device fixed on a small platform, and checked whether there were corresponding products available for purchase and use through the selection of products on the market. 


4. According to the fiber laser used, the relevant parameters of the laser fiber collimator was determined by theoretical calculation. Then, the product selection was conducted according to the parameters. 

5. Circuit design.According to the requirements, a audio acquisition circuit was designed, which was controlled by STM32 single chip microcomputer to store and operate the audio data.Designed the voltage level matching circuit between STM32 single chip microcomputer (CMOS level) and MEMS reverse optical modulator (TTL level), and used Multisim for circuit simulation to determine feasibility and reliability.STM32 microcontroller decoded the data received by the detector and analyzed the audio data contained in it.Conducted circuit diagram drawing through Altium Designer. 

6. 3D modeling with SolidWorks and CAD, and Conducted engineering drawings, including assembly drawing of active end, optical path drawing and program flow chart of the whole reverse modulation system.


%---------------------------------------------------------------------------%
\englishkeywords{MEMS MRR;FSO;RC Cassegrain;ZEMAX;Multisum;Solidworks}% 英文关键词
%---------------------------------------------------------------------------%
