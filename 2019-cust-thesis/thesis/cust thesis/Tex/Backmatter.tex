\chapter[致谢]{致\quad 谢}\chaptermark{致\quad 谢}% syntax: \chapter[目录]{标题}\chaptermark{页眉}
\thispagestyle{noheaderstyle}% 如果需要移除当前页的页眉
\pagestyle{noheaderstyle}% 如果需要移除整章的页眉

经过了一个学期的实践,在孟立新老师的指导下,我最终完成了《基于MEMS的逆向光通信系统》的光机电设计。
    
长春理工大学在国内外光学研究处于前沿,学校最初由王大珩院士创立,在光机电系统方面有着深厚的底蕴,我的专业是机械电子专业,当初我初衷就是为了以后有机会能参与光机电系统项目设计,所以我选择《基于MEMS的逆向光通信系统》这个毕设题目,感谢孟立新老师给了我这个机会,让我能够有机会能在南区的空地激光通信国防重点学科实验室做实验和毕设。

这是我第一次接触这样的研究类课程,这一学期系统地学习了项目从开题到结题整个过程的研究方法,学会了各类报告的书写,更加熟悉了STM32单片机的使用、Soldworks、Altium Designer、Multisim、Citavi、ZEMAX、KEIL、LaTeX等软件的使用。
为了更好的设计光机结构,我在寒假学习了《应用光学》、《现代光学设计方法》和《光机系统设计》相关的课程和书籍。学习了相关的光学软件,包括ZEMAX、Light Tools和Code V。学习体会最深的是ZEMAX,在理论学习之后,通过ZEMAX软件建模,通过优化后,查询点列图等评价工具可以很容易评估光学系统的性能,从而及时纠正自己的理论设计,加深了我对理论知识的理解。虽然通过这次毕设实践,我系统学习了很多光学知识,开阔了我的认识和拓展了我的思维,但是我知道我只是接触了光学领域的冰山一角,还有很多光学知识需要学习和掌握。

在电路设计,以前从来没有接触过激光调制解调电路方面的知识,通过查询书籍和孟老师的指导,我最终学会了基于OOK调制方式的激光发射程序的编写,也最终想出了光电探测器后端的信号解调电路的设计。除此之外,我将这学期学会的STM32单片机操作技能,如SD卡读写、FATFS文件系统移植、触摸屏和音频解调芯片的使用等功能融合,最终设计出了基于SMT32单片机的触摸屏人机交互界面,通过触屏和按键操作可以实现音乐播放、录制语音、音频信息的激光调制和解调操作。

在机械设计方面,我对传统激光通信系统主动端的光机结构设计有了深刻的认识,孟老师对我最初的卡式光学系统的机械结构提出了很多建议,让我受益良多。通过这次毕设的光机结构设计,我对机械与光学两大部分是如何进行合作和光机设计有了系统的认识。

在本文的撰写过程中,孟立新老师作为我的指导老师,从论文选题到搜集资料,从开题报告、写初稿到反复修改,给予了我很多的帮助,给了提供了很多宝贵的建议。

在光机系统设计过程中,孟老师结合其自身的工程实践经验,对我设计的卡塞格林的光机系统从材料选择到结构优化提出很多修改建议,让我受益匪浅。

我通过这次课题《基于MEMS逆向光通信系统的设计》的研究和学习,了解并熟悉光机结构系统的设计流程,对激光通信有了更深的认识,对研究课题的方法步骤有深刻的体会和认识,在这次学习过程之后,无论是从理论知识(光学,光机设计,光通信),还是工程设计方面(Solidworks,Zemax,CAD,Altium designer),有了更深的领悟和体会。

感谢赵洪刚学长给我提供了很多帮助,在毕设中给我提供了很多阅读材料,使我能快速找到研究方向,并带领我初步做了实验,大致了解了实验流程,使我在后面资料收集中更得心应手,也给我解答了很多问题。

最后我十分感谢孟立新老师在毕业设计过程中,给予我极大的帮助,使我对整个毕设的思路有了总体的把握,孟老师很耐心的帮我解决了很多实际的问题,使我有了很大的收获,孟老师在整个设计过程中提出了很多建设性的建议,并给我解决了一些专业性问题。

%\cleardoublepage[plain]% 让文档总是结束于偶数页,可根据需要设定页眉页脚样式,如 [noheaderstyle]
