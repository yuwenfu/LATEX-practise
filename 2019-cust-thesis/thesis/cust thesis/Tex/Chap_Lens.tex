\chapter{光学系统}\label{chap:Lens Design}
\section{反射式物镜设计}
反射式望远镜物镜在空间光学系统中有着广泛的应用,因此无论是在国内还是国外它都成了一个研究热点。对于空间光学系统,由于其物距非常大,而探测器的像元尺寸有限,如果要取得一定的分辨率,就需要增大系统的焦距,通常空间光学系统的焦距都会在几百毫米以上,长的可以达到数米甚至数十米。由于焦距长,要达到一定的相对孔径,物镜的口径就显得非常大,可以达到几百毫米至数米。这样大的口径对于透射式系统来说,是非常难以实现的,因此通常空间光学系统都采用反射式。另外,反射式系统的另一个优点是没有色散,适用于宽光谱系统。反射式光学系统通常有两镜式或多镜式。
\section{一、两镜系统设计}
两镜系统由一个主镜和一个次镜组成,通常主镜和次镜都是二次曲面,其表达式为
$$ y^2 = 2rx-(1-e^2 )x^2 $$
其中$ e^2  $面形参数,可以作为消像差的自变量;r为镜面顶点的曲率半径。
对于望远镜系统,其物体位于无限远,同时一般光阑与主镜重合,因此有
$$ l_1=\infty,u_1=0 $$
定义两个与外形尺寸有关的参数
$$ \alpha = \dfrac{l_2}{f_{1}^{'}} = \dfrac{2l_2}{r_1} \approx \dfrac{h_2}{h_1}$$
$$ \beta = \dfrac{l_2 ^{}}{f_{2}^{'}} =  \dfrac{u_2}{u_2^{'}}$$
根据高斯公式,还可以写出
$$ r_{2} =  \dfrac{\alpha \cdot\beta}{1+\beta}\cdot r_1$$
其中,$ \alpha $ 表示次镜离第一焦点的距离,也决定了次镜的遮光比;$ \beta $表示次镜的放大倍数。主镜的焦距乘以$ \beta $即为系统的焦距,或主镜的F数乘以$ \beta $的绝对值即为系统的F数。
两镜系统的最大优点是主镜的口径可以做得较大,远超过透镜的极限尺寸,镀反射膜后,使用波段很宽,没有色散,同时采用非球面后,有较大的消像差的能力。因此,两镜嗯,系统结构比较简单,成像质量优良。但是,两镜系统也有一些缺点,例如不容易得到较大成像质量所需的视场,次镜会引起中心遮拦,有时遮拦还比较大,非球面与球面相比,制造难度更大,但现在非球面加工技术越来越成熟,因此在空间光学系统中,两镜系统仍然是一个很好的选择。
\section{天文望远镜R-C系统设计}
首先由仪器的总体设计要求,确定光学系统的通光口径及总的相对孔径。主镜的相对孔径的选择和多方面因素有关,在经典的卡塞格林及R-C系统中,主要和系统的相对孔径有关。若系统的焦距比较长,主镜的焦距可以长一些,相对孔径也就可以小一些,这样加工容易一些。若系统的焦距很短,则主镜焦距就必须取得较短,相对孔径变大,从缩短镜筒长度来说,主镜相对孔径越大越有利,但加工难度会相应增大,加工难度和相对孔径立方成正比。因此,主镜相对孔径数值的确定需要综合几方面的因素来定,一般取1:3左右。
另一个问题就是确定焦点的伸出量$ \Delta $,在消像差的独立变量中,与外形尺寸有关的是$ \alpha $和$ \beta $。当$ \Delta $值较大,又要维持一定的$ \beta $值不太大,势必要增大$ \alpha $值,从而中心遮拦增大。$ \alpha $、$ \beta $、$ \Delta $之间的关系为
\begin{equation*}
\left\{ 
\begin{array}{l}  
	l_2 = \dfrac{-f_{1}^{'}+\Delta}{\beta -1}\cdot r_1 
	\\
	\alpha = \dfrac{l_2}{f_1 ^{'}}
\end{array}
\right.  
\end{equation*}
主镜和次镜之间的间隔以及次镜的半径为
\begin{equation*}
\left\{  \begin{array}{l}   d = f_1^{'}(1-\alpha) \\
	 r_2 = \dfrac{\alpha \cdot \beta}{\beta + 1}*r_1 \end{array} \right.
\end{equation*}
主镜的半径为
$$r_1 = 2 \times \dfrac{\text{主镜口径}}{\text{主镜的相对孔径}}$$

\section{卫星R-C系统设计}
假设需要设计一个用于空间卫星的R-C系统,系统口径为250mm,系统的焦距为1000mm,系统伸长量为180mm,要求镜头长度尽可能短。
因为整个系统的相对孔径比较大,为1:4,以假定主镜的相对孔径为1:2.  这样主镜的焦距为-500mm,顶点曲率半径为-1000mm,从主镜到系统焦点的距离为500+180=680(mm),因此
$$ r_1=-1000 $$
$$ f_1 ^{'}=-1000 $$
$$
\beta=\dfrac{1000}{-500}=-2
$$
次镜的放大率为2($\beta = - 2$),故次镜离主镜焦点的距离为
$$ l _ { 2 } = \frac { - f _ { 1 } ^ { \prime } + \Delta } { \beta - 1 } = \frac { - ( - 500 ) + 180 } { - 2 - 1 } = \frac { 680 } { - 3 } = - 226.667 $$
而
$$ \alpha = \frac { l _ { 2 } } { f _ { 1 } ^ { \prime } } = \frac { - 226.667 } { - 500 } = 0.4533333 $$
同时根据消球差和慧差的条件,有
$$ \mathrm { e } _ { 1 } ^ { 2 } = 1 + \frac { 2 \alpha } { ( 1 - \alpha ) \beta ^ { 2 } } = 1 + \frac { 2 \times 0.453 } { ( 1 - 0.4533333 ) \times ( - 2 ) ^ { 2 } } = 1.4146341 $$
$$ e _ { 2 } ^ { 2 } = \frac { \frac { 2 \beta } { 1 - \alpha } + ( 1 + \beta ) ( 1 - \beta ) ^ { 2 } } { ( 1 + \beta ) ^ { 3 } } = \frac { \frac { 2 \times ( - 2 ) } { 1 - 0.4533333 } + ( 1 - 2 ) ( 1 + 2 ) ^ { 2 } } { ( 1 - 2 ) ^ { 3 } } = \frac { - 7.3170727 - 9 } { - 1 }=16.317073 $$
由于卫星外形尺寸的限制,希望镜筒尽量短一些,次镜遮拦少些,现在$\alpha = 0.453$,中心遮拦损失达20.6%,主镜和次镜之间的距离达-500 + 226.667 = -273.333(mm)。经再三验算,将主镜相对孔径提高到1:1.2,即主镜焦距取-300mm,但同样的工程可以求出
$$ \begin{array} { l } { \alpha = 0.3696667 , \beta = - 3.3332632 } \\ { e _ { 1 } ^ { 2 } = 1.1055676 , e _ { 2 } ^ { 2 } = 4.2816786 } \\ { r _ { 1 } = - 600 \mathrm { mm } , r _ { 2 } = - 316.86 \mathrm { mm } } \\ { d = - 189.10 \mathrm { mm } } \end{array} $$
将所有参数输入Zemax软件,取视场角为0.1°,系统图如图所示。