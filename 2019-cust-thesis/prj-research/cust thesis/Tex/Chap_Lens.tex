\chapter{光学系统}\label{chap:Lens Design}
\section{反射式物镜设计}
反射式望远镜物镜在空间光学系统中有着广泛的应用,因此无论是在国内还是国外它都成了一个研究热点。对于空间光学系统,由于其物距非常大,而探测器的像元尺寸有限,如果要取得一定的分辨率,就需要增大系统的焦距,通常空间光学系统的焦距都会在几百毫米以上,长的可以达到数米甚至数十米。由于焦距长,要达到一定的相对孔径,物镜的口径就显得非常大,可以达到几百毫米至数米。这样大的口径对于透射式系统来说,是非常难以实现的,因此通常空间光学系统都采用反射式。另外,反射式系统的另一个优点是没有色散,适用于宽光谱系统。反射式光学系统通常有两镜式或多镜式。