%-*- coding: UTF-8 -*-
%2.1.1.2 正确使用标点

\documentclass[UTF8]{ctexart}
%P68 txfonts增加了数学字体的支持,是效果最好的免费字体
\usepackage{txfonts}	
%高德纳的Concrete正文字体和Zapf的Euler数学字体配合需要综合使用字体包cccfonts和euler,这一组合字体清晰适合PPT演示使用
\usepackage[T1]{fontenc} %使用新的宏包fontenc来选择字体的编码,XELATEX使用的是Unicode编码(在NFSS中一般是EU1),传统的LATEX字体编码有一般正文字体OT1,扩展字体T1,数学字母OML,数学符号OMS,数学扩展符号OMX等。需要手工设置的通常只有正文字体的编码,默认的正文编码是OT1,而扩展码T1则包含更多的符号,特别是带重音的拉丁字母等,许多字体包都要求使用T1编码,字体说明文档会有说明
\usepackage{ccfonts,eulervm}	

\begin{document}

“\,'A' or 'B?'\,”he asked.

X-ray

1--2

dash---like

one...

one two\ldots

one two three\dots

\# \quad \$ \quad \% \quad \& \quad
\{ \quad \}	\quad \_ \quad
\textbackslash	%"\"

\punctstyle{quanjiao}全角式,所有标点全角,有挤压。%全角
\punctstyle{banjiao}半角式,所有标点半角,有挤压。

Mr.~K	%带子“~”,TEX禁止在这种空格之间分行,因而用来表示一些不宜分开的情况

%LATEX把大写字母后的点看做缩写标记,把小写字母后的点看做是句子结束,并对他们使用不同的间距
%使用\ 表明使用普通的空格,或用\@.指明.是大写字母后的句末
Roman number XII. Yes

Roman number XII\@. Yes

条目-a不同于条目-b

\mbox{条目}-a 不同于条目-b %忽略汉字与其他内容之间的空格

\CJKsetecglue{}	%手动设置汉字与其他内容之间的内容为空
汉字word

幻影\phantom{参数}速速隐形	%产生与参数内容一样大小的空格

\hphantom{123}幻影参数速速隐形\vphantom{123}

速速\vphantom{12}幻影

这是一行文字\\另一行

这是一行文字\linebreak 另一行	%\linebreak 一行散开分布

\symbol{28450}\symbol{35486}
\symbol{"DE}\symbol{"FE}

%P63
{\bfseries \slshape Bold front test}	%粗体倾斜
\par 	%分段
\sffamily	%字体族,字体形状,字体系列
\textbf{This is a paragraph of bold and
\textit{italic front,sometimes returning to \textnormal{normal front} is necessary. }}

{\itshape M}M \quad {\itshape M\/}M	%\/倾斜修正 

%NFSS为字体划分了编码,族,系列,形状,尺寸等多个正交属性
%需要用\selectfont命令使它们生效
%P67
\fontencoding{OT1}\fontfamily{pzc}
\fontseries{mb}\fontshape{it}	%it是意大利斜体的缩写
\fontsize{18}{17}\selectfont
PostScript New Century Schoolbook

%另一方法
\usefont{T1}{pbk}{db}{n}
PostScript Bookman Demibold Normal



\end{document}

