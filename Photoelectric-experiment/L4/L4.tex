\documentclass{zjureport}
% =============================================
% Part 1 Edit the info
% =============================================

%%%%%%%%%%%%%%
\makeatletter
\newcommand\dlmu[2][4cm]{\hskip1pt\underline{\hb@xt@ #1{\hss#2\hss}}\hskip3pt}
\makeatother
%%%%%%%%%%%%%%

\newcommand{\major}{机械电子工程}
\newcommand{\name}{付煜文}
\newcommand{\stuid}{150321128}
\newcommand{\newdate}{2018年第4次}
\newcommand{\loc}{机办301}

\newcommand{\course}{光电检测实验}
\newcommand{\tutor}{丁红昌}
\newcommand{\grades}{}
\newcommand{\newtitle}{光纤位移传感器实验}
\newcommand{\exptype}{验证性实验}
\newcommand{\group}{None}

% =============================================
% Part 1 Main document
% =============================================
\begin{document}
		\thispagestyle{plain}         %只保留页脚
	\begin{figure}[h]
		\begin{minipage}{0.6\linewidth}
			\centerline{\includegraphics[width=\linewidth]{custed.png}}
		\end{minipage}
		\hfill
		\begin{minipage}{0.4\linewidth}
			\raggedleft
			\begin{tabular*}{.8\linewidth}{ll}
				专业: & \underline\major   \\
				姓名: & \underline\name    \\
				学号: & \underline\stuid   \\
				日期: & \underline\newdate \\
				地点: & \underline\loc
			\end{tabular*}
		\end{minipage}
	\end{figure}
	
	\begin{table}[!htbp]
		\centering
		\begin{tabular*}{\linewidth}{llllll}
			课程名称: & \underline\course   & 指导老师: & \underline\tutor   & 成绩:       &  \dlmu[1.5cm]{} \\
			实验名称: & \underline\newtitle & 实验类型: & \underline\exptype 
		\end{tabular*}
	\end{table}

% =============================================
% Part 2 Main document
% =============================================

\section{实验目的和要求}
	{了解光纤位移传感器的工作原理和性能}
\section{基本原理}
	{
		本实验采用的传光型光纤,它由两束光纤混合后组成Y 型光纤,半圆分布即双D 型一束光纤
		端部与光源相接发射光束,另一束端部与光电转换器相接接收光束。两光束混合后的端部是工作
		端亦称探头,它与被测体相距X,由光源发出的光传到端部出射后再经被测体反射回来,由另一束
		光纤接收光信号经光电转换器转换成电量,而光电转换器的电量大小与间距X 有关,因此可用于
		测量位移。}
\section{主要仪器设备}
  {主机、Y 型光纤传感器、光纤支架及紧固螺钉、测微头、反射面}

\section{操作方法和实验步骤}
	\addimagebig{L4-1-yuanli}{光纤传感器安装示意图}
	\addimagebig{L4-2-yuanli}{光纤传感器位移实验接线图}
 	\begin{enumerate}
 		\item{
 			根据图 11—1 安装 Y 型光纤位移传感器,光纤二根尾纤分别插入实验模板上的光电变换
 			座中。其内部已和发光管 D 及光电转换管 T 相接。} 
 		\item{
 			按图 11-2 接线,将光纤实验电路输出端VO1 与主机的电压表相连。} 
 		\item{
	 		合上主机电源开关,调节测微头使反射面与光纤传感器相连。调节 Rw1,使电压表(量程为
	 		20V)显示为零。} 
 		\item{
 			旋转测微头,反射面离开探头方向,每隔 0.1mm 读出数显表值,将其填入表\ref{tb1}}
 		
 		\begin{table}[!htbp]
 			\centering
 			\caption{}\label{tb1}
 			\begin{tabular}{c|cccccccccc}
 				\Xhline{1pt}
 				位移X(mm) &0 &0.1 &0.2 &0.3	 &0.4 &0.5 & 0.6 & 0.7\\
 				\Xhline{0.4pt}
 				电压U(V) & 0 & 0.013 & 0.093 & 0.176 & 0.268 & 0.35 & 0.464 & 0.612\\
 				\Xhline{1pt}
 			\end{tabular} 
 		\end{table}
 		测量的时候注意回程差,即不要小范围来回调节位移,最好顺着单方向调节位移 

 		\item{根据上表,作光纤位移传感器的位移特性曲线,计算在量程 1mm 时灵敏度和非线性误差。} 
 	\end{enumerate}
 
	\addimagesml{L4-1-data}{光纤位移传感器的位移特性曲线}
	\addimagebig{L4-2-data}{非线性度误差计算}
	
	\newpage
	$$\xi_{L}=\pm \dfrac{\Delta L_{max}}{y_{FS}}\times 100\% =\pm \dfrac{0.0718}{0.9} =14.36\% $$
	
\section{思考题}

	{光纤位移传感器测位移时对被测物体的表面有些什么要求?}
	
	\section*{字符}
	
	\makebox[0.15\textwidth][l]{测试}{测试1\hfill{测试2}}
	
	\section*{字符}
	\nomenclatureitem[\textbf{Unit}]{\textbf{Symbol}}{\textbf{Description}}
	\nomenclatureitem[$\Unit{m^{2} \cdot s^{-2} \cdot K^{-1}}$]{$R$}{the gas constant}
	\nomenclatureitem[$\Unit{m^{2} \cdot s^{-2} \cdot K^{-1}}$]{$C_v$}{specific heat capacity at constant volume}
	\nomenclatureitem[$\Unit{m^{2} \cdot s^{-2} \cdot K^{-1}}$]{$C_p$}{specific heat capacity at constant pressure}
	\nomenclatureitem[$\Unit{m^{2} \cdot s^{-2}}$]{$E$}{specific total energy}
	\nomenclatureitem[$\Unit{m^{2} \cdot s^{-2}}$]{$e$}{specific internal energy}
	\nomenclatureitem[$\Unit{m^{2} \cdot s^{-2}}$]{$h_T$}{specific total enthalpy}
	\nomenclatureitem[$\Unit{m^{2} \cdot s^{-2}}$]{$h$}{specific enthalpy}
	\nomenclatureitem[$\Unit{kg \cdot m \cdot s^{-3} \cdot K^{-1}}$]{$k$}{thermal conductivity}
	\nomenclatureitem[$\Unit{kg \cdot m^{-1} \cdot s^{-2}}$]{$S_{ij}$}{deviatoric stress tensor}
	\nomenclatureitem[$\Unit{kg \cdot m^{-1} \cdot s^{-2}}$]{$\tau_{ij}$}{viscous stress tensor}
	\nomenclatureitem[$\Unit{1}$]{$\delta_{ij}$}{Kronecker tensor}
	\nomenclatureitem[$\Unit{1}$]{$I_{ij}$}{identity tensor}
	
	
	
	

\end{document}
