\documentclass{zjureport}
% =============================================
% Part 1 Edit the info
% =============================================

%%%%%%%%%%%%%%
\makeatletter
\newcommand\dlmu[2][4cm]{\hskip1pt\underline{\hb@xt@ #1{\hss#2\hss}}\hskip3pt}
\makeatother
%%%%%%%%%%%%%%

\newcommand{\major}{机械电子工程}
\newcommand{\name}{付煜文}
\newcommand{\stuid}{150321128}
\newcommand{\newdate}{2018年第1次}
\newcommand{\loc}{机办301}

\newcommand{\course}{光电检测实验}
\newcommand{\tutor}{丁红昌}
\newcommand{\grades}{}
\newcommand{\newtitle}{光 源 和 光 的 分 光 实 验}
\newcommand{\exptype}{验证性实验}
\newcommand{\group}{None}

% =============================================
% Part 1 Main document
% =============================================
\begin{document}

		\thispagestyle{plain}         %只保留页脚

\begin{figure}[h]
  \begin{minipage}{0.6\linewidth}
    \centerline{\includegraphics[width=\linewidth]{custed.png}}
  \end{minipage}
  \hfill
  \begin{minipage}{.4\linewidth}
    \raggedleft
    \begin{tabular*}{.8\linewidth}{ll}
      专业: & \underline\major   \\
      姓名: & \underline\name    \\
      学号: & \underline\stuid   \\
      日期: & \underline\newdate \\
      地点: & \underline\loc
    \end{tabular*}
  \end{minipage}
\end{figure}

\begin{table}[!htbp]
  \centering
  \begin{tabular*}{\linewidth}{llllll}
    课程名称: & \underline\course   & 指导老师: & \underline\tutor   & 成绩:       &  \dlmu[1.5cm]{} \\
    实验名称: & \underline\newtitle & 实验类型: & \underline\exptype 
  \end{tabular*}
\end{table}
% =============================================
% Part 2 Main document
% =============================================

\section{实验目的和要求}
	{通过实验使学生对光源,光源分光原理、光的不同波长等基本概念有具体认识。}
\section{基本原理}
	{本实验中备有普通光源和激光光源。普通光源(白炽灯)光谱为连续光谱( 白炽灯的另一个
		特性是做灯丝的钨有正阻特性,工作时的热电阻远大于冷态时的电阻,在灯的启动瞬时有较大的
		电流 )。 利用分光三棱镜后,可以提供红色,黄色,绿色,蓝色等多种波长的光辐射。激光光源
		是半导体激光器,发射出波长为630纳米的红色光(激光特性:\textcircled{1}单色性\textcircled{2}方向性\textcircled{3}相干性等)。}

\section{主要仪器设备}
  {主机、普通光源、分光装置(三棱镜)、半导体激光器。}

\section{操作方法和实验步骤}

	\addimage{L1-1-fenguang}{分光实验}
	
	\begin{enumerate}
		\item{
			根据图\ref{L1-1-fenguang} 进行组装和接线,用实验线将主机中 AC12V 交流电源输出与普通光源相连接。合上主机的总电源开关。}
		\item{
			松开图\ref{L1-1-fenguang}中光源或三棱镜的升降固定螺钉,调节高度使光束对准三棱镜,转动三棱镜座使三棱镜毛面在后面,二个工作面(光面)的棱在前面。然后调节涡杆角度使折射的投射面(狭缝端盖)上出现清晰的光谱。如果光谱不清晰可轻微旋转光源罩(灯丝方向)和松开升降杆固定螺钉转动一个角度(光束方向)使光束对准三棱镜的工作面(要点:光束对准棱镜工作面﹑灯丝方向)。}
		\item{
			闭主机总电源开关。 将图\ref{L1-1-fenguang}中的普通光源取下, 换上半导体激光源(旋下前端盖小孔),将激光源与主机激光电源相应连接﹙注意颜色-极性﹚。打开主机总电源开关,根据步骤 2 调节观察投射面现象(单色性)。
					}
	\end{enumerate}

\section{思考题}
	\begin{enumerate}
	\item{解释实验现象。}
	\item{半导体激光器的特性有哪些?半导体激光器的发散角一般为5$\degree$ ~10 $\degree$,你如何利用实验装置和直尺完成最简易的发散角测量实验方法。}
	\end{enumerate}

\end{document}
