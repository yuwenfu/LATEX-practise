\documentclass{zjureport}
% =============================================
% Part 1 Edit the info
% =============================================

%%%%%%%%%%%%%%
\makeatletter
\newcommand\dlmu[2][4cm]{\hskip1pt\underline{\hb@xt@ #1{\hss#2\hss}}\hskip3pt}
\makeatother
%%%%%%%%%%%%%%

\newcommand{\major}{机械电子工程}
\newcommand{\name}{付煜文}
\newcommand{\stuid}{150321128}
\newcommand{\newdate}{2018年第3次}
\newcommand{\loc}{机办301}

\newcommand{\course}{光电检测实验}
\newcommand{\tutor}{丁红昌}
\newcommand{\grades}{}
\newcommand{\newtitle}{光敏三极管特性实验}
\newcommand{\exptype}{验证性实验}
\newcommand{\group}{None}

\renewcommand\labelenumi{(\theenumi)} %重新定义编号,如(x)

% =============================================
% Part 1 Main document
% =============================================
\begin{document}
		\thispagestyle{plain}         %只保留页脚
	\begin{figure}[h]
		\begin{minipage}{0.6\linewidth}
			\centerline{\includegraphics[width=\linewidth]{custed.png}}
		\end{minipage}
		\hfill
		\begin{minipage}{.4\linewidth}
			\raggedleft
			\begin{tabular*}{.8\linewidth}{ll}
				专业: & \underline\major   \\
				姓名: & \underline\name    \\
				学号: & \underline\stuid   \\
				日期: & \underline\newdate \\
				地点: & \underline\loc
			\end{tabular*}
		\end{minipage}
	\end{figure}
	
	\begin{table}[!htbp]
		\centering
		\begin{tabular*}{\linewidth}{llllll}
			课程名称: & \underline\course   & 指导老师: & \underline\tutor   & 成绩:       &  \dlmu[1.5cm]{} \\
			实验名称: & \underline\newtitle & 实验类型: & \underline\exptype 
		\end{tabular*}
	\end{table}

% =============================================
% Part 2 Main document
% =============================================

\section{实验目的和要求}
	{了解光敏三极管结构、性能和V-I 特性。}
\section{基本原理}
	{
		在光敏二极管的基础上,为了获得内增益,就利用晶体三极管的电流放大作用,用Ge 或Si
		单晶体制造NPN 或PNP 型光敏三极管。其结构使用电路及等效电路如图\ref{L3-1-yuanli}所示。}
	
	\addimagebig{L3-1-yuanli}{光敏三极管结构及等效电路}
		
	{光敏三极管可以等效一个光电二极管与另一个一般晶体管基极集电极并联 :集电极-基极产
		生的电流,输入到共发三极管的基极在放大。不同之处是,集电极电流(光电流)有集电结上产
		生的i$_{\phi}$控制。集电极起双重作用;把光信号变成电信号起光电二极管作用;使光电流再放大起一
		般三极管的集电结作用。一般光敏三极管只引出E、C 两个电极,体积小,光电特性是非线性的,
		广泛应用于光电自动控制作光电开关应用。}	
	
\section{主要仪器设备}
  {主机、光敏三极管、光源、照度计及探头、分光装置}

\section{操作方法和实验步骤}
  \subsection{光敏三极管伏安特性}
	光敏三极管在不同的照度下的伏安特性就象一般晶体管在不同的基极电流输出特性一样。光
	敏三极管把光信号变成电信号。
	\addimagebig{L3-2-jiexian}{光敏二极管实验接线图}
	\begin{enumerate}
		\item{
			将图\ref{L3-2-jiexian}中的光敏二极管换成光敏三极管,按图接线(光敏三极管探头红色接电压表示图
			的+,黑色接电流表示图的+)注意探头颜色, 主机的电流表的量程在实验过程中需要进行切换,从$\mu$A 到 mA 档,电压表的量程为 20V 档。} 
		\item{
			首先缓慢调节 0~20mA 电流源(光源电压),使光源的光照度在某一照度值(2、4、6、8
			Lx),再调节主机 0-5v 电源改变光敏三极管的电压,测量光敏三极管的输出电流和电压。填入表\ref{tb1},并作出一定光照度下的光敏三极管的伏安特性曲线(可多做几组族线)} 
	\end{enumerate}
	
	\begin{table}[!htbp]
		\centering
		\caption{}\label{tb1}
		\begin{tabular}{c|c|l|ccccccccccc}
			\Xhline{1pt}
			\multicolumn{2}{l|}{光敏三极管伏安特性}	 & 	电压U(V) & 0 & 0.5 & 1 & 1.5 & 2 & 2.5 & 3 & 3.5 & 4 & 4.5 & 5\\
			\Xhline{0.4pt}
			\multirow{4}{*}{照度(Lx)}& 2Lx & 电流I(uA) & 0 & 3.9	4 & 4 & 4.1 & 4.1 & 4.1 & 4.2 & 4.2	 & 4.2 & 4.3\\
			\Xcline{2-14}{0.4pt}
			& 4Lx	 & 电流I(uA & 0 & 8.2	 & 8.3 & 	8.4	 & 8.5 & 	8.6	 & 8.7	 & 8.7	 & 8.8 & 	8.8	 & 8.9\\
			\Xcline{2-14}{0.4pt}
			& 6Lx	 & 电流I(uA) & 0 & 16	 & 16.2	 & 16.4	 & 16.5 & 	16.6 & 16.7	 & 16.8 & 	16.9 & 17 & 17\\
			\Xcline{2-14}{0.4pt}
			& 8Lx	 & 电流I(uA) & 0 & 21.8 & 22.1 & 22.3	 & 22.6 & 	22.8 & 23 & 23.1 & 23.3	 & 23.5	 & 23.6\\
			\Xhline{1pt}
		\end{tabular} 
	\end{table}
	\addimagebig{L3-3-data}{光敏三极管的伏安特性}
  \subsection{光敏三极管的光照特性测量}
	{实验方法同实验三(参照实验三中的 1.光照特性的测试)。将实验数据填入表\ref{tb2},并作出
		I-Lx 特性曲线。}
	\begin{table}[!htbp]
		\centering
		\caption{}\label{tb2}
		\begin{tabular}{c|ccccccccccc}
			\Xhline{1pt}
			照度(Lx) & 5 & 10 & 15 & 20 & 25& 30 & 35 & 40 & 	50& 60\\
			\Xhline{0.4pt}
			电流I(uA) & 14.04 & 30 & 47.3 & 59.4 & 80.6 & 105.1 & 113.8 & 135.2 & 166.7 & 210\\
			\Xhline{1pt}
		\end{tabular} 
	\end{table}
 
	\addimagebig{L3-4-data}{光敏三极管的光照特性}

\section{思考题}
	{光敏二极管、光敏三极管的应用场合?}


\end{document}
