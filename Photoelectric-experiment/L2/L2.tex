\documentclass{zjureport}
% =============================================
% Part 1 Edit the info
% =============================================

%%%%%%%%%%%%%%
\makeatletter
\newcommand\dlmu[2][4cm]{\hskip1pt\underline{\hb@xt@ #1{\hss#2\hss}}\hskip3pt}
\makeatother
%%%%%%%%%%%%%%

\newcommand{\major}{机械电子工程}
\newcommand{\name}{付煜文}
\newcommand{\stuid}{150321128}
\newcommand{\newdate}{2018年第2次}
\newcommand{\loc}{机办301}

\newcommand{\course}{光电检测实验}
\newcommand{\tutor}{丁红昌}
\newcommand{\grades}{}
\newcommand{\newtitle}{光敏电阻实验}
\newcommand{\exptype}{验证性实验}
\newcommand{\group}{None}

\renewcommand\labelenumi{(\theenumi)} %重新定义编号,如(x)

% =============================================
% Part 1 Main document
% =============================================
\begin{document}
		\thispagestyle{plain}         %只保留页脚
	\begin{figure}[h]
		\begin{minipage}{0.6\linewidth}
			\centerline{\includegraphics[width=\linewidth]{custed.png}}
		\end{minipage}
		\hfill
		\begin{minipage}{.4\linewidth}
			\raggedleft
			\begin{tabular*}{.8\linewidth}{ll}
				专业: & \underline\major   \\
				姓名: & \underline\name    \\
				学号: & \underline\stuid   \\
				日期: & \underline\newdate \\
				地点: & \underline\loc
			\end{tabular*}
		\end{minipage}
	\end{figure}
	
	\begin{table}[!htbp]
		\centering
		\begin{tabular*}{\linewidth}{llllll}
			课程名称: & \underline\course   & 指导老师: & \underline\tutor   & 成绩:       &  \dlmu[1.5cm]{} \\
			实验名称: & \underline\newtitle & 实验类型: & \underline\exptype 
		\end{tabular*}
	\end{table}


% =============================================
% Part 2 Main document
% =============================================

\section{实验目的和要求}
	{了解光敏电阻的光照特性、光谱特性和伏安特性等基本特性。}
\section{基本原理}
	{
		在光线的作用下,电子吸收光子的能量从键合状态过渡到自由状态,引起电导率的变化,这
		种现象称为光电导效应。光电导效应是半导体材料的一种效应。光照愈强,器件自身的电阻愈小。
		基于这种效应的光电器件称光敏电阻。光敏电阻无极性,其工作特性与入射光光强、波长和外加
		电压有关。}


\section{主要仪器设备}
  {主机、安装架、发光二极管光源、光敏电阻探头、
  	光照度计及探头、分光装置。}

\section{操作方法和实验步骤}
  \subsection{亮电阻和暗电阻测量}
	\addimagesml{L2-1-guangmin}{光敏器件实验原理图}
	\addimagebig{L2-2-jiexian}{光照度计测量接线}
	\addimagebig{L2-3-jiexian}{光敏器件实验安装接线图}
	\begin{enumerate}
		\item{图\ref{L2-1-guangmin}是光敏电阻实验原理图}
		\item{按图\ref{L2-2-jiexian}光照度实验安装接线。将照度计探头与主
			机小面板上照度计显示表 Vi 口相连接(照度计探头的黄
			色接照度表的 Vi,黑色接地)。按图\ref{L2-2-jiexian}接。打开主机
			电源,然后,顺时针慢慢调节 0$\sim$20mA 可调电流源旋钮,使照度计显示为 100Lx。}
		\item{
			撤下照度计探头,换上光敏电阻探头及电路(图\ref{L2-3-jiexian}顺时针慢慢调节 0 $\sim$ 5V 可调电源,
			使电压表显示 5V(如调不到 5V 则 Vcc 改接 0 -- 15V 可调电压源)。}
		\item{
			在光敏电阻与光源之间用遮光筒连接,10 秒钟后,读取电压表(量程为 20V 档)和电流表量程为 20mA 档)的值分别为亮电压$U_{\text{亮}}$和亮电流$I_{\text{亮}}$。}
		\item{将 0$\sim$20mA 可调电流源的调节旋钮逆时针方向慢慢旋到底, 10 秒钟后,读取电压表(量程
			为 20V 档)和电流表(量程为 20µA 档)的值分别为暗电压$U_{\text{暗}}$和暗电流$I_{\text{暗}}$。}
		\item{根据以下公式,计算亮阻和暗阻}
		\par
	$R_{\text{亮}}$=$U_{\text{亮}}$/$I_{\text{亮}}$ = $\dfrac{4.92 V}{1.74 mA }$= 2.82k$\Omega$ 
 \qquad $R_{\text{暗}}$=$U_{\text{暗}}$/$I_{\text{暗}}$=$\dfrac{5.73 V}{0.03 \mu A} $ = 191M$\Omega$ 
		\item{光敏电阻在不同的照度下有不同的亮阻和暗阻; 在不同的工作电压下有不同的亮阻和暗阻。
			如有兴趣可重复以上实验步骤做实验。}
	\end{enumerate}
  \subsection{光照特性测量}
	当光敏电阻的工作电压(Vcc)为+5V 时,光敏电阻的光电流随光照强度变化而变化,它们之
	间的关系是非线性的。改变光源电流大小可得到不同的光照度值(实验方法同以上实验,照度计探
	头和光敏电阻探头交替使用),测得数据填入表\ref{tb1},并作出光电流与光照度 I-Lx 曲线图。
	\begin{table}[!htbp]
		\centering
		\caption{}\label{tb1}
		\begin{tabular}{lccccccccc}
			%\toprule
			\Xhline{1pt}
			光照度(L$_X$) & 20 & 40 & 60 & 80 & 100 & 120 & 140 & 160 & 180  \\
			%\cmidrule
			%\midrule
			\Xhline{0.4pt}
			电流(mA) & 0.37 & 0.62 & 0.92 & 1.18 & 1.49 & 1.92 & 2.12 & 3.48 & 3.60  \\
			%\bottomrule
			\Xhline{1pt}
		\end{tabular}
	\end{table}

	\addimagebig{L2-1-data.pdf}{光照特性测量}
	
  \subsection{伏安特性测量}
  在一定的光照强度下,光敏电阻的光电流随外加电压的变化而变化,实验时,在给定光照强
  度为 50Lx、100Lx、150Lx 时,图\ref{L2-2-jiexian}改变光敏电阻的工作电压值$\Delta$ U=0.5(由电压表监测),测
  得不同光照度下流过光敏电阻的电流值,将数据填入表\ref{tb2}, 并作不同照度下的三条伏安特性曲线。
\begin{table}[!htbp]
  	\centering
  	\caption{}\label{tb2}
	\begin{tabular}{c|l|l|ccccccccccc}
	\Xhline{1pt}
	\multicolumn{2}{c|}{型号:G5528} & 电压(U) & 0	& 0.50 & 1 & 1.50 & 2 & 2.50 & 3 & 3.50	 & 4 & 4.50 & 5\\
	\Xhline{0.4pt}
	\multirow{3}{*}{照度(Lx)} & 50 & 电流(mA) & 0 & 0.07 & 0.15 & 0.23 & 0.31	 & 0.39	 & 0.47 & 0.54	 & 0.59	 & 0.68	 & 0.81\\
	\Xcline{2-14}{0.4pt}
	 & 100 & 电流(mA) & 0 & 0.18 & 0.34 & 0.53 & 0.71 & 0.87 & 1.05 & 1.23 & 1.41 & 1.58 & 1.76\\
	\Xcline{2-14}{0.4pt}
	& 150 & 电流(mA) & 0 & 0.35	 & 0.70 & 1.04 & 1.45 & 1.76 & 2.10 & 2.46 & 2.8 & 3.14 & 3.51\\
	\Xhline{1pt}
\end{tabular} 
\end{table}

	\addimagebig{L2-2-data.pdf}{伏安特性测量}
	
\section{思考题}
	{
		为什么测光敏电阻亮阻和暗阻要经过10 秒钟后读数,这是光敏电阻的缺点,只能应用于什么
		状态?}

\end{document}
